

%%%%%%%%%%%%%%%%%%%%%%%%%%%%%%%%%%%%%%%%%%%%%%%%%%%%%%%%%%%%%%%%%%%%%%%%%%%%%
%                                                                           %
%          LaTeX File for Doctor (Master) Thesis of ECNU                    %
%            华东师范大学博士(硕士)论文模板 ____lizb                      %
%                                                                           %
%%%%%%%%%%%%%%%%%%%%%%%%%%%%%%%%%%%%%%%%%%%%%%%%%%%%%%%%%%%%%%%%%%%%%%%%%%%%%


%NOTE: build options in latex : Options -> build -> XeLatex

\documentclass[12pt,openany,a4paper,fancyhdr,twoside]{ctexbook}

%draft 选项可以使插入的图形只显示外框,以加快预览速度。
%s\documentclass[12pt,openany,a4paper,fancyhdr,oneside,draft]{ctexbook}

\usepackage[CJKbookmarks,breaklinks]{hyperref}
\usepackage{shortvrb,indentfirst,makeidx}
\usepackage[normalem]{ulem}
\usepackage{fancyhdr}
\usepackage{graphicx}
\usepackage{indentfirst,latexsym,amsthm,colortbl,subfigure,clrscode}
%\usepackage{algorithm}
%\usepackage[noend]{algorithmic}
\usepackage[ruled, vlined, linesnumbered]{algorithm2e}
\usepackage{bm}                     % 处理数学公式中的黑斜体的宏包
\usepackage{amsmath}                % AMSLaTeX宏包 用来排出更加漂亮的公式
\usepackage{amssymb}                % AMSLaTeX宏包 用来排出更加漂亮的公式
\usepackage{mathrsfs}
\usepackage[subnum]{cases}
\usepackage[numbers,sort&compress]{natbib} 
\usepackage{hypernat}
\usepackage{geometry}
\usepackage{multirow}
\usepackage{times}
\usepackage{fontspec}
\usepackage{libertine}
\usepackage{url}
%break line for long urls
\def\UrlBreaks{\do\/\do-}
\usepackage{caption}
\usepackage{titletoc}
\usepackage{slashbox}
\usepackage{array}
\usepackage{indentfirst} 
\usepackage{color}
\usepackage{epstopdf}
\usepackage{rotating}
\usepackage{supertabular} %supertabular表格宏包
\usepackage{colortbl} %彩色表格宏包
\usepackage{booktabs} %表格线宏包
\usepackage{longtable} 
\usepackage{bigstrut,multirow,rotating}
\usepackage{enumerate}
\usepackage{xspace}
\usepackage{listings}
\usepackage{framed}
\usepackage{makecell}
\usepackage{pifont}
\usepackage{listings}
\usepackage{clrscode}
\usepackage{xcolor}
\usepackage{float}
\usepackage{pdfpages}

\renewcommand\thefootnote{\ding{\numexpr171+\value{footnote}}}

%设置bib的行间距
\usepackage{setspace}
%设置章节第一页的顶边距
%\usepackage{titlesec}
%\usepackage{etoolbox}
\makeindex
\pagestyle{fancy}
\newcommand{\comment}[1]{}
\renewcommand{\headrulewidth}{0.5pt}



\newcommand{\loflabel}{图}%在图目录编号前添加Tab.
\newcommand{\lotlabel}{表}

\newcommand{\ie}{\textit{i.e., }}
\newcommand{\aka}{\textit{i.e., }}
\newcommand{\eg}{\textit{e.g., }}
\newcommand{\etal}{\textit{et al. }}

\newcommand{\tabincell}[2]{\begin{tabular}{@{}#1@{}}#2\end{tabular}}


\lstset{
	numbers=left, 
	language=[ANSI]{C++},
	numberstyle= \small, 
	keywordstyle= \color{ blue!100},
	commentstyle= \color{gray}, 
	flexiblecolumns, 
	frame=single, 
	rulesepcolor= \color{ red!20!green!20!blue!20} ,
	%escapeinside=``,
	xleftmargin=2em,xrightmargin=0em, aboveskip=2em,
	framexleftmargin=2em,
	basicstyle = \ttfamily\fontsize{10}{11}\selectfont,
	otherkeywords = {},
	emph={parallel_for,Partition,Build,Probe},
	emphstyle=\color{red},
	captionpos=b
}


%                    根据自己正文需要做的一些定义                 %
%==================================================================
\def\diag{{\rm diag}}
\def\rank{{\rm rank}}
\def\RR{{\cal R}}
\def\NN{{\cal N}}
\def\R{{\mathbb R}}
\def\C{{\mathbb C}}
\let\dis=\displaystyle

\def\p{\partial}
\def\f{\frac}
\def\mr{\mathrm}
\def\mb{\mathbf}
\def\mc{\mathcal}
\def\b{\begin}
\def\e{\end}

\renewcommand{\algorithmcfname}{算法}
\renewcommand{\lstlistingname}{清单}
%\renewcommand{\algorithmautorefname}{算法}
\SetKwRepeat{Do}{do}{while}
\newtheorem{thm1}{Theorem}[part]
\newtheorem{thm2}{Theorem}[section]
\newtheorem{thm3}{Theorem}[subsection]
\newtheorem{them}[thm2]{定理}
\newtheorem{theorem}[thm2]{定理}
\newtheorem{defn}[thm2]{定义}
\newtheorem{define}[thm2]{定义}
\newtheorem{ex}[thm2]{例}
\newtheorem{exs}[thm2]{例}
\newtheorem{example}[thm2]{例}
\newtheorem{prop}[thm2]{命题}
\newtheorem{lemma}[thm2]{引理}
\newtheorem{cor}[thm2]{推论}
\newtheorem{remark}[thm2]{注释}
\newtheorem{notation}[thm2]{记号}
\newtheorem{despription}[thm2]{描述}
\newtheorem{abbre}[thm2]{缩写}
\newtheorem{problem}[thm2]{问题}
%重新设置证明的其实标题

%\floatname{algorithm}{算法}
%\renewcommand{\proofname}{\textit{证明}}
%\usepackage{etoolbox}  % patch def of algorithmic environment
%\makeatletter
%\patchcmd{\algorithmic}{\addtolength{\ALC@tlm}{\leftmargin} }{\addtolength{\ALC@tlm}{\leftmargin}}{}{}
%\makeatother
%\renewcommand\algorithmiccomment[1]{%
%	\hfill\#\ #1%
%}

\newcommand{\yihao}{\fontsize{26pt}{36pt}\selectfont}           % 一号, 1.4 倍行距
\newcommand{\erhao}{\fontsize{22pt}{28pt}\selectfont}          % 二号, 1.25倍行距
\newcommand{\xiaoer}{\fontsize{18pt}{18pt}\selectfont}          % 小二, 单倍行距
\newcommand{\sanhao}{\fontsize{16pt}{24pt}\selectfont}        % 三号, 1.5倍行距
\newcommand{\xiaosan}{\fontsize{15pt}{22pt}\selectfont}        % 小三, 1.5倍行距
\newcommand{\sihao}{\fontsize{14pt}{21pt}\selectfont}            % 四号, 1.5 倍行距
\newcommand{\banxiaosi}{\fontsize{13pt}{19.5pt}\selectfont}    % 半小四, 1.5倍行距
\newcommand{\xiaosi}{\fontsize{12pt}{18pt}\selectfont}            % 小四, 1.5倍行距
%\newcommand{\xiaosi}{\fontsize{12pt}{12pt}\selectfont}            % 小四, 1倍行距
\newcommand{\dawuhao}{\fontsize{11pt}{11pt}\selectfont}       % 大五号, 单倍行距
\newcommand{\wuhao}{\fontsize{10.5pt}{15.75pt}\selectfont}    % 五号, 单倍行距


%============================ 可以自定义文字块 ================================%

\newcommand{\aaa}{Example}
\newcommand{\bbb}{\aaa \aaa \aaa}
\newcommand{\ccc}{\bbb \bbb \bbb \bbb \bbb
	
	\bbb \bbb \bbb \bbb \bbb }
\newcommand{\abc}{abcdefg1234567890}
\newcommand{\upabc}{ABCDEFGHIJK}

\newcommand{\reminder}[1]{ {\mbox{$<==$}} [[[ { \bf #1 } ]]] {\mbox{$==>$}}}

%%% ----------------------------------------------------------------------
%学科代码:0835Z1
%设置每章标题前后的间距
\CTEXsetup[beforeskip = 0pt]{chapter}
\CTEXsetup[afterskip = 20pt]{chapter}
\setlength{\baselineskip}{25pt}  %%正文设为25磅行间距

%============================= 版芯控制 ================================%
\setlength{\oddsidemargin}{0.57cm} 
\setlength{\evensidemargin}{\oddsidemargin}
\voffset-6mm \textwidth=150mm \textheight=230mm \headwidth=150mm
%\rightmargin=35mm
%                                                                       %

%============================= 页面设置 ================================%
%-------------------- 定义页眉和页脚 使用fancyhdr 宏包 -----------------%
% 定义页眉与正文间双隔线
%\newcommand{\makeheadrule}{%
%	\makebox[0pt][l]{\rule[.7\baselineskip]{\headwidth}{0.4pt}}%
%	\rule[0.85\baselineskip]{\headwidth}{0.4pt} \vskip-.8\baselineskip}
\renewcommand{\headrulewidth}{0.5pt}
\fancyhf{}
\fancyhead[CO,CE]{华东师范大学硕士专业学位研究生学位论文}
% % 设置奇数页的页脚
% \fancyfoot[RO]{\thepage}
% % 设置偶数页的页脚
% \fancyfoot[LE]{\thepage}
% 设置页脚中的页码居中
\fancyfoot[C]{\small\thepage}
\makeatletter

%新加的,为每章首页添加页眉
\makeatletter
\let\ps@plain\ps@fancy
\makeatother

%\renewcommand{\headrule}{%
%	{\if@fancyplain\let\headrulewidth\plainheadrulewidth\fi
%		\makeheadrule}} \makeatother

\newcommand{\adots}{\mathinner{\mkern 2mu%
		\raisebox{0.1em}{.}\mkern 2mu\raisebox{0.4em}{.}%
		\mkern2mu\raisebox{0.7em}{.}\mkern 1mu}}

\setmainfont{Times New Roman}
\dottedcontents{chapter}[1.5cm]{\xiaosi\heiti}{3.8em}{9.5pt}
\dottedcontents{section}[1.5cm]{\xiaosi\heiti}{2.8em}{9.5pt}

\newcommand{\thesisTitleZh}{论文标题具体内容}
\newcommand{\thesisTitleEn}{thesisTitleEn}
\newcommand{\authorname}{***}
\newcommand{\director}{***}
\newcommand{\school}{数据科学与工程学院}
\newcommand{\authornameEn}{***}
\newcommand{\directorEn}{*****}
\newcommand{\schoolEn}{School of Data Science and Engineering}
\newcommand{\stuid}{******}
\newcommand{\TODO}[1]{\textit{\textcolor{red}{[TODO: #1]}}}
%\newcommand{\TODO}[1]{}
\newcommand{\keyidea}[1]{\textbf{\textcolor{blue}{[#1]}}}
%\newcommand{\keyidea}[1]{}
\newcommand{\blankSmall}{\vspace{-0.2cm}}
\newcommand{\blankLarge}{\vspace{-0.30cm}}
\newcommand{\lst}[1]{清单~{#1}}
\newcommand{\pstart}[1]{\textbf{#1:}}
\SetKwComment{Comment}{ $\triangleright$\ }{}
\newcommand\mycommfont[1]{\small\itshape{#1}}
\setlength{\textfloatsep}{10pt} 
\newcommand{\hide}[1]{#1}
\newcommand{\eat}[1]{}
%\newcommand{\PrintVersion}[1]{#1}
\newcommand{\PrintVersion}[1]{}
%=============================== 正文部分 ================================%
\lstdefinestyle{myCustomMatlabStyle}{
	language={[ISO]C++},
	numbers=left,
	%	float=tp!,
	stepnumber=1,
	numberstyle=\ttfamily\small,
	tabsize=4,
	keepspaces=true, %
	%breakindent=22pt, %
	showspaces=false,
	showstringspaces=false,
	%stringstyle=\ttfamily,
	keepspaces=true,
	%escapebegin=\begin{CJK*}{GBK}{hei},escapeend=\end{CJK*},
	basicstyle=\small,
	breaklines=true,
	flexiblecolumns=true, %% font
	breakatwhitespace=false,
	fontadjust,
	captionpos=b,
	framextopmargin=1pt,framexbottommargin=1pt,
	abovecaptionskip=3pt,belowcaptionskip=3pt,
	texcl=true,	
	xleftmargin=2em,xrightmargin=0em,
	frame=single,
	extendedchars=false,columns=flexible,mathescape=true,
	escapechar=@,
	morecomment=[l]{//},
	aboveskip=0em, 
	breakindent=0pt
}

\renewcommand{\contentsname}{\textbf{\zihao{-3}\heiti 目\ 录}}
\renewcommand{\listfigurename}{\textbf{\zihao{-3}\heiti 插\ 图\ 目\ 录}}
\renewcommand{\listtablename}{\textbf{\zihao{-3}\heiti 表\ 格\ 目\ 录}}

\begin{document}
\pagestyle{empty}
\setlength{\baselineskip}{25pt}  %%正文设为25磅行间距
\vspace{-2.0cm}
\noindent{{\zihao{4} {\large 2025} 届硕士专业学位研究生学位论文}}\\
\vspace{-0.8cm}
\begin{flushleft}
\hspace{-0.5cm}
\renewcommand\arraystretch{1.5}
\begin{tabular}{l}
\noindent{{\zihao{4} 分类号:\underline{~~~~~~~~~~~~~~~~~~~~~~~~}}}  \\ 
\noindent{{\zihao{4} 密~~~~级:\underline{~~~~~~~~~~~~~~~~~~~~~~~~}}}\\ 
\end{tabular}
\hskip 3.2cm
\renewcommand\arraystretch{1.5}
\begin{tabular}{l}
\noindent{{\zihao{4} 学校代码:\underline{~~~~~~~~10269~~~~~~~~}}}\\ 
\noindent{{\zihao{4} 学~~~~~~~~号:
\underline{~~~~~~~~~~~~~~~~~~~~~~~~~~}}}\\ 
% \underline{~~\hide{\stuid}~~}}}\\ 
\end{tabular}
\end{flushleft}


\vskip 1.8cm

\begin{center}
\hskip 0.5cm
  \scalebox{1.0}{\includegraphics[width=2.6cm]{fig/ecnu_logo.png}}
\hspace{0.3cm}
\scalebox{1.0}{\includegraphics[width=10.5cm]{fig/ecnulabel.png}}
\vskip 0.5cm
{\textbf{{\xiaoer East China Normal University}}}\\ \vskip 0.2cm
{\textbf{\erhao 硕~士~专~业~学~位~论~文}}\\ \vskip 0.2cm
{\textbf{{\xiaoer Master's Degree Thesis (Professional)}}}\\
\end{center}

\vskip 0.8cm

\begin{center}

{\erhao \bf 论文题目:\underline{论文题目具体内容}}
\end{center}

\vskip 0.8cm 
\begin{center}

\renewcommand\arraystretch{1.5}
	\begin{tabular}{l}
		\eat{{\sihao \bf 院\qquad\ \ \ 系:}\\ }
{\sihao \bf 院~系~名~称:}\\ 
{\sihao \bf 专业学位类别:}\\ 
{\sihao \bf 专业学位领域:}\\
{\sihao \bf 研~究~方~向:}\\ 
{\sihao \bf 指~导~教~师:}\\ 
{\sihao \bf 学位申请人:}
\end{tabular}
\begin{tabular}c
{\sihao \bf  ~~\school}               \\ 
\hline {\sihao \bf ~~电子信息 }              \\
\hline {\sihao \bf ~~大数据技术与工程 }              \\ 
\hline {\sihao \bf ~~******~~}\\ 
\hline {\sihao \bf ~~ \ \ \hide{\director~}\hide{职称},***~职称}  \\
\hline{\sihao \bf  ~~ \hide{\authorname}}      \\ \hline
\end{tabular}
\end{center}
\vskip 2.0cm
\begin{center}
{\sihao 2025年**月}
\end{center}
%额外空白页
%\clearpage
%\phantom{s}
%\clearpage

\PrintVersion{
	\clearpage
	\phantom{s}
	\clearpage
}
\newpage

\pagestyle{empty}

\noindent{\large Thesis for Master's Degree (Professional) in 2025}\\
\begin{flushright}
 {\large University Code: 10269}\\	
 {\large Student ID: \hide{\stuid}}
\end{flushright}



\vskip 2cm

\begin{center}
{\Huge $\mathbb{EAST}\,\mathbb{CHINA}\,\mathbb{NORMAL}\,
\mathbb{UNIVERSITY}$}
\end{center}

\vskip 3cm

\begin{center}
\bfseries{\scshape{\huge \thesisTitleEn}}\\
\end{center}

\vskip 2.5cm {\large
\begin{center}
\begin{tabular}{l}
Department:\\
Category:\\
Domain:\\
Research direction:\\
Supervisor:\\
Candidate:
\end{tabular}
\begin{tabular}c
~~~\schoolEn  \\
\hline ~~~Electronic Information\\
\hline ~~~Big Data Technology and Engineering\\
\hline ~~~***********\\ 
\hline ~~~***********\\ 
% \hline ~~~\hide{Prof.****}  \hide{\directorEn}, A.P.~*****\\
\hline ~~~  \hide{\authornameEn}\\
\hline
\end{tabular}
\end{center}}

\vskip 25mm

\begin{center}
{\sihao 2025}
% {\Large Month, 2024}
\end{center}
%额外空白页
%\clearpage
%\phantom{s}
%\clearpage

\PrintVersion{
	\clearpage
	\phantom{s}
	\clearpage
}
\newpage
\pagestyle{empty}
\centerline{\bf\Large 华东师范大学学位论文原创性声明}

\vskip 1cm

\normalsize \indent
郑重声明:本人呈交的学位论文《 \thesisTitleZh 》,是在华东师范大学攻读硕士/博士(请勾选)学位期间,在导师的指导下进行的研究工作及取得的研究成果。除文中已经注明引用的内容外,本论文不包含其他个人已经发表或撰写过的研究成果。对本文的研究做出重要贡献的个人和集体,均已在文中作了明确说明并表示谢意。
$$\\  $$

\qquad\qquad{作者签名}:$\underline{\qquad\qquad\qquad }$
\qquad \qquad\qquad \mbox {日期}:\qquad 年 \qquad  月 \qquad  日


\vskip 1cm

\centerline{\bf\Large 华东师范大学学位论文著作权使用声明}

\vskip 1cm

《 \thesisTitleZh 》系本人在华东师范大学攻读学位期间在导师指导下完成的硕士/博士(请勾选)学位论文,本论文的研究成果归华东师范大学所有。本人同意华东师范大学根据相关规定保留和使用此学位论文,并向主管部门和相关机构如国家图书馆、中信所和“知网”送交学位论文的印刷版和电子版;允许学位论文进入华东师范大学图书馆及数据库被查阅、借阅;同意学校将学位论文加入全国博士、硕士学位论文共建单位数据库进行检索,将学位论文的标题和摘要汇编出版,采用影印、缩印或者其它方式合理复制学位论文。

本学位论文属于(请勾选)

(  )1.经华东师范大学相关部门审查核定的“内部”或“涉密”学位论文*,
于     年    月    日解密,解密后适用上述授权。

(  )2.不保密,适用上述授权。
$$\\ $$
\qquad\qquad \mbox{导师签名}:$\underline{\qquad\qquad\qquad\qquad}$
\qquad\qquad \mbox {本人签名}:$\underline{\qquad\qquad\qquad\qquad }$

\vskip 0.5cm

$\rightline{ \qquad 年 \qquad  月 \qquad  日 \qquad\qquad}$

%这里改成了0.5,之前是1
\vskip 0.2cm

* “涉密”学位论文应是已经华东师范大学学位评定委员会办公室或保密委员会审定过的学位论文(需附获批的《华东师范大学研究生申请学位论文“涉密”审批表》方为有效),未经上述部门审定的学位论文均为公开学位论文。此声明栏不填写的,默认为公开学位论文,均适用上述授权)。
%额外空白页
%\clearpage
%\phantom{s}
%\clearpage

\PrintVersion{
	\clearpage
	\phantom{s}
	\clearpage
}
\newpage
\pagestyle{empty}
$$\\ \\ \\ $$

\centerline{\bf\Large $\underline{\mbox{\kaishu{\authorname}}\,\textbf{硕士}}$学位论文导师指导小组声明}

\vskip 18mm
本人已认真阅读此篇学位论文。我认为,该论文已达到申请硕士学位的要求,可启动学位申请程序。
\begin{flushright}
    签名: \qquad \qquad \qquad \qquad \qquad \qquad \qquad \qquad
\end{flushright}
\begin{flushright}
    XX,导师,电子信息(大数据技术与工程)
\end{flushright}
\begin{flushright}
    ****年**月**日
\end{flushright}

\vskip 18mm
本人已认真阅读此篇学位论文。我认为,该论文已达到申请硕士学位的要求,可启动学位申请程序。
\begin{flushright}
    签名: \qquad \qquad \qquad \qquad \qquad \qquad \qquad \qquad
\end{flushright}
\begin{flushright}
    XX,副导师,电子信息(大数据技术与工程)
\end{flushright}
\begin{flushright}
    ****年**月**日
\end{flushright}

\vskip 18mm
本人已认真阅读此篇学位论文。我认为,该论文已达到申请硕士学位的要求,可启动学位申请程序。
\begin{flushright}
    签名: \qquad \qquad \qquad \qquad \qquad \qquad \qquad \qquad
\end{flushright}
\begin{flushright}
    XX,电子信息(大数据技术与工程)
\end{flushright}
\begin{flushright}
    ****年**月**日
\end{flushright}

%额外空白页
\clearpage
此处扫描页:开题签名单
%\phantom{s}
\clearpage

\PrintVersion{
	\clearpage
	\phantom{s}
	\clearpage
}

\newpage
\pagestyle{empty}
$$\\ \\ \\ $$


\centerline{\bf\Large $\underline{\mbox{\kaishu{\authorname}}}\,\,
	$硕士学位论文答辩委员会成员名单}

\vskip 10mm

	\begin{center}
	{\large
	\begin{tabular}{| p{25mm}| p{20mm}| p{40mm}| p{35mm}|}
		\hline
	\vfill\hfill{\heiti 姓名}\hspace*{\fill} &
	\vfill\hfill{\heiti 职称}\hspace*{\fill} &
	\vfill\hfill{\heiti 单位}\hspace*{\fill} &
	\vfill\hfill {\heiti 备注\xspace\xspace\xspace\xspace\xspace\xspace\xspace\xspace\xspace\xspace} \hspace*{\fill} \\[6pt]
	\hline
	
	\vfill\hfill{xxx}\hspace*{\fill} &
	\vfill\hfill{xxx}\hspace*{\fill} &
	\vfill\hfill{xxx}\hspace*{\fill} & 
	\vfill\hfill{主席\xspace\xspace\xspace\xspace\xspace\xspace\xspace\xspace\xspace\xspace}\hspace*{\fill} \\[6pt]\hline
	
	\vfill\hfill{xxx}\hspace*{\fill} &
	\vfill\hfill{xxx}\hspace*{\fill} &
	\vfill\hfill{xxx}\hspace*{\fill} & 
	\vfill\hfill {\xspace}\hspace*{\fill} \\[6pt]\hline
	
	\vfill\hfill{xxx}\hspace*{\fill} &
	\vfill\hfill{xxx}\hspace*{\fill} &
	\vfill\hfill{xxx}\hspace*{\fill} & 
	\vfill\hfill {\xspace}\hspace*{\fill} \\[6pt]\hline
	
	\vfill\hfill{xxx}\hspace*{\fill} &
	\vfill\hfill{xxx}\hspace*{\fill} &
	\vfill\hfill{xxx}\hspace*{\fill} & 
	\vfill\hfill {\xspace}\hspace*{\fill} \\[6pt]\hline
	
	\vfill\hfill{xxx}\hspace*{\fill} &
	\vfill\hfill{xxx}\hspace*{\fill} &
	\vfill\hfill{xxx}\hspace*{\fill} & 
	\vfill\hfill {\xspace}\hspace*{\fill} \\[6pt]\hline
	
	\end{tabular}
	}
	\end{center}



%额外空白页
%\clearpage
%\phantom{s}
%\clearpage

\PrintVersion{
	\clearpage
	\phantom{s}
	\clearpage
}

%摘要
\newpage
\pagenumbering{Roman}
\pagestyle{plain}
\chapter*{\zihao{-3}\heiti{摘~~~~要}}
摘要正文内容...

如果你需要分点的话:


\begin{enumerate}[~(1)]
	\item \pstart{分点1}内容。
	
	\item \pstart{分点2}内容。
\end{enumerate} 


{\sihao{\textbf{关键词:}}} xxx,xxx
\addcontentsline{toc}{chapter}{摘要}
\newpage
\vspace{-2.5cm}
\chapter*{\zihao{-3}\heiti{ABSTRACT}}

\vspace{-0.5cm}


\setlength{\baselineskip}{17pt}
content...




\begin{enumerate}[~(1)]
	\item \pstart{part1}content.
	
	\item \pstart{part2} content.
	
\end{enumerate} 



%\hspace{-0.5cm}
{\sihao{\textbf{Keywords:}}} \textit{xxx, xxxx}


%\clearpage
%\phantom{s}
%\clearpage
\addcontentsline{toc}{chapter}{ABSTRACT}
\clearpage
\phantom{s}
\clearpage



\setcounter{secnumdepth}{3}

\setcounter{tocdepth}{2}
\tableofcontents


\clearpage
\phantom{s}
\clearpage

% 插图目录
\renewcommand{\numberline}[1]
{\loflabel~#1\hspace*{1em}}%设置图目录编号样式,主要是间距
\listoffigures%自动生成图目录

% 表格目录
\renewcommand{\numberline}[1]{\lotlabel~#1\hspace*{1em}}
\listoftables
\clearpage
\phantom{s}
\clearpage



\newpage
\pagenumbering{arabic}
\pagestyle{plain}

\CTEXsetup[format+={\zihao{3}\heiti}]{chapter}
\CTEXsetup[format+={\raggedright\zihao{4}\heiti}]{section}
\CTEXsetup[format+={\zihao{-4}\heiti}]{subsection}
%\CTEXsetup[format+={\fontsize{10.5pt}{15.75pt}\heiti}]{subsection}

\setlength{\baselineskip}{25pt}  %%正文设为25磅行间距
%\setlength{\baselineskip}{20pt}  %%正文设为25磅行间距


\chapter{绪论} \label{chap:introduction}
图形学物理模拟是计算机图形学中的一个重要研究方向,其核心目标是通过计算机模拟物理现象,生成真实感强的动态图像和动画效果。随着计算机硬件的不断进步和图形处理能力的提升,图形学物理模拟在多个领域的应用逐渐拓展,尤其在游戏开发、电影制作、虚拟现实(VR)和增强现实(AR)等行业中发挥着关键作用。在游戏开发中,物理模拟涉及碰撞检测、实时流体解算、实时布料解算、实时刚体解算、角色布娃娃系统等,极大提升了游戏的沉浸感和互动性。在电影制作中,物理模拟帮助特效团队生成复杂的自然现象,如爆炸、火焰、烟雾和水流等,增强了视觉效果的真实性。此外,物理模拟还在工程仿真、医学模拟、教育培训等领域得到了广泛应用,为各行各业提供了更为精准的分析和预测工具。现代应用程序中的许多技术依赖于固体与流体之间的动态交互,例如充气橡胶轮胎、液压系统、飞行器、漂浮的船只、风车等。随着虚拟手术、数字化制造和软体机器人等新兴应用的出现,流固耦合的快速计算方法的需求也日益增加。这些方法不仅能够提升虚拟环境中的真实感,还能在许多实际应用中提供高效的解决方案,尤其是在需要精确模拟和实时交互的场合。随着相关技术的不断发展,如何实现高效、精准且具有高逼真度的流固耦合仿真,成为了当前物理模拟领域中的一项重要挑战。本章节将从研究背景,研究现状,研究目标这三个方面对物理仿真中的流固耦合和物质点法进行全面的介绍,之后总结了本文的研究目标与贡献,最后给出了全文的组织结构。

\section{研究背景与意义}\label{sect:intro:bg}
在物理模拟中,拉格朗日视角(Lagrangian perspective)和欧拉视角(Eulerian perspective)是两种常见的描述物理现象的方法,它们在处理流体动力学和固体力学等问题时具有各自的优势和应用场景。拉格朗日视角以物体的运动为核心,物体被离散为很多个物质点,每个物质点代表了物体在某一小空间范围内的物质属性,在算法流程中关注物质点随时间的轨迹和变形。通过跟踪每个物质点的运动状态,拉格朗日视角能够直接描述物体内部的变形、应力和速度等物理量。欧拉视角则固定一个观察区域,并关注物体在该区域内的性质变化。在欧拉方法中,计算网格是固定的,物质随着时间流动经过网格,模拟的焦点是物质通过固定空间的运动和变形。拉格朗日视角可以轻松追踪每个物质点的运动轨迹,因此在模拟大尺度的软体形变和流体拓扑变化方面有较大优势,但在计算物理量梯度时需要搜集邻居物质点的信息,计算开销比较昂贵;欧拉视角由于计算网格固定,通过有限差分法可以方便计算物理量梯度,但在描述物体拓扑变化方面比较困难。因此通过网格法做出的流体模拟效果往往在视觉表现上更像比较黏稠的果冻。混合拉格朗日/欧拉方法可以综合两者的优点,在网格上计算物理量梯度,把计算结果插值回粒子后,继续追踪粒子的位置变化。

\begin{figure}[htbp]
    \centering
    \includegraphics[width=1\linewidth]{fig/mpm_cases.png}
    \caption{物质点法的模拟场景}
    \label{fig:mpm-cases}
\end{figure}

在实践中,往往根据物体材料的特性选用不用的模拟视角。例如,对于烟雾模拟常常采用欧拉视角进行模拟,对空间区域进行网格划分,追踪每个网格烟雾密度的变化;对于拓扑变化频繁的液体,在仿真过程中,其形态可能在液块、液面、液滴之间切换,所以常常用拉格朗日视角方法,比如:光滑粒子流体动力学方法(SPH)或者混合拉格朗日/欧拉方法,比如:粒子网格法(PIC)、流体隐式粒子法(FLIP);对于可变形固体,往往采用欧拉视角的可变形网格方法,比如有限元方法(FEM),或者混合拉格朗日/欧拉方法的物质点法(MPM),它们都基于本构模型对材料物理方程的弱形式进行求解。

近年来,物质点法(MPM)得到了大量研究,在模拟真实世界中取得了前所未有的效果。物质点法是一种混合拉格朗日/欧拉方法,可以作为流体、气体、弹性固体、刚体等材料的统一模拟框架,同时也支持和其他欧拉或拉格朗日方法进行耦合。物质点法具备以下优点:其一,物质点法可以从动量守恒的弱形式推导出来,从而实现物理定律的精确离散化;其二,由于网格和粒子间的相互插值,边界条件、材料碰撞等模拟中需要特别关注的部分可以不用额外代价、自然地被解决;其三,通过赋予粒子不同的材料属性或本构模型,可以轻松实现自动的多材料和多相耦合。如今,物质点法已经可以应用在绝大多数场景,模拟大量不同的材料,包括水、蜂蜜、牛奶、非牛顿流体、沙子、雪、布料、果冻、橡胶、汽车、肉类、木头、海绵等等。图~\ref{fig:mpm-cases}展示了物质点法模拟各种真实场景的强大能力。

\begin{figure}[htbp]
    \centering
    \includegraphics[width=1\linewidth]{fig/houdini.png}
    \caption{节点物理仿真软件Houdini的操作界面}
    \label{fig:houdini}
\end{figure}

然而,现有的物质点法在算法层面和工程角度仍有优化空间。一方面,物质点法算法本身在模拟物体碰撞、流固耦合时会产生数值耗散,造成物理非真实感的缺陷。此外,因为碰撞的判定是基于网格的,受网格分辨率影像,在物体碰撞边界处会出现空白缝隙这一非真实效果。另一方面,在国内游戏影视和工业生产领域,缺少类似于国外开发的Houdini这种基于节点系统的物理仿真软件。节点系统是国外顶级影视特效团队多年工作中打磨出来的现代化特效生产工作流,可以视作一种可视化编程方法,节点连线便于特效师、动画师等非程序开发者进行艺术效果打磨;节点本身可以在程序测由专业的图形程序来实现。图~\ref{fig:houdini}展示了Houdini的节点系统示意图和效果。

因此,本文将基于物质点法求解框架,继续研究算法优化细节和配套工程。随着国内游戏和影视领域越来越追求高品质动画、高真实感沉浸式体验,如何利用物质点法打磨更精细的物理动画效果以及加速特效领域工业流水线的建设成为了难题,面临的主要挑战有:

\begin{enumerate}
    \item[1.] 在物质点法算法流程中涉及两次粒子网格的信息交换,包括粒子到网格(P2G)和网格到粒子(G2P),由于粒子数量一般多于网格点数量,因此导致了自由度的丢失,在模拟液体时会出现数值粘性问题,这个问题在模拟流固耦合中尤其严重,常常出现非粘性的流体经过软体表面时被固体粒子捕获住难以顺畅流过和分离的现象。
    \item[2.] 在物质点法中,两个粒子如果映射到同一个网格点上,就会判定为接触,进而计算相互的物理影响,因此在解决自碰撞问题时,天然存在一个网格边长的分离距离,导致在网格分辨率不高时,相互碰撞的物体可以肉眼观察到一条空白的缝隙。
    \item[3.] 影视特效行业有迫切使用物质点法的需求来制作各种材料物体的交互,但国产市场缺少现代化的基于节点的物理仿真软件。节点化系统通过将算法流程的特定步骤抽象为节点,可以同其他节点进行上下游的通信,例如将物质点法中的粒子到网格传输(P2G)这一步抽象成节点,特效师和算法研究团队就可以定制工作流和打磨最终产品,例如在P2G这一步后对网格进行风格化编辑,添加一个新的节点,引入额外力场,打磨艺术化效果。
\end{enumerate}


针对上述问题和挑战,本文旨在优化物质点法的内在缺陷,通过改造粒子网格间的传输方式解决非物理数值粘性问题,通过自适应网格划分更进一步节省计算开销并优化自碰撞边界空白缝隙问题,并设计与开发一套自研节点仿真系统实现本文提出的改进MPM算法,并在大规模流固耦合仿真场景中进行测试,做出算法效率、算法效果与制作流程便利性上的分析。

\section{国内外研究现状}
\subsection{流固耦合仿真}
在计算机图形学领域,流固耦合现象的仿真一直是研究的热点,固体模拟和流体模拟通常采用不同的离散方法和数值方法,如何正确处理交接处的数值解算是研究的难点。
Batty等~\cite{batty2007fast} 通过将压力求解表示为能量最小化问题,开发了一种方法,用于在笛卡尔欧拉流体与具有不规则几何形状的拉格朗日刚体之间进行双向耦合。Wolper等~\cite{klingner2006fluid} 提出了使用非结构化动态四面体网格来模拟流体的方法,该方法可以与拉格朗日刚体耦合。Chentanez~\cite{chentanez2006simultaneous} 公式化了一个单一的非对称线性系统,用于模拟流体与可变形固体之间的相互作用;该方法后来被 Zarifi~\cite{zarifi2017positive} 扩展为求解一个对称正定(SPD)系统。基于拉格朗日粒子的方法也能实现稳定的流固耦合。Akinci等~\cite{akinci2012versatile} 提出了一个基于流体动力学力的SPH流体与任意刚体耦合方法,该方法后来被扩展以支持弹性固体~\cite{akinci2013coupling}。

自FLIP方法被Zhu~\cite{zhu2005animating}引入图形学界以来,关于混合粒子-网格方法的研究取得了显著进展。Gao等~\cite{gao2009simulating} 提出了一个混合算法,通过结合网格和粒子方法高效地模拟气态流体,能够捕捉低速和高速行为。Losasso等~\cite{losasso2008two} 提出了一个双向耦合仿真框架,结合了用于密液体的粒子水平集方法和用于弥漫区的新型SPH方法,实现了密液体和弥漫区水体的高效建模,能够实现无缝混合和交互。Raveendran等~\cite{raveendran2011hybrid} 提出了一个混合方法,用于在SPH中强制保持不可压缩性。

\subsection{物质点法}
MPM是一种混合方法,通过使用拉格朗日粒子来承载材料信息,并使用欧拉网格进行力计算,从而统一了固体和流体的处理方法。最初为计算机图形学中的雪模拟所引入~\cite{stomakhin2013material},后来,MPM被扩展到处理多材料仿真中的相变问题~\cite{stomakhin2014augmented}。MPM能够处理多种材料,包括弹性-粘塑性体和非牛顿流体。Jiang等~\cite{jiang2015affine}提出了仿射粒子-网格转移方案,相比FLIP方法,减少了能量耗散,并提供了更高的稳定性。Klar等~\cite{klar2016drucker}在MPM中离散化了Drucker-Prager塑性模型来模拟沙流,并进一步扩展到沙水耦合模拟~\cite{tampubolon2017multi}。Jiang等~\cite{jiang2017anisotropic}还修改了MPM工作流,以支持布料模拟。Gao等~\cite{gao2017adaptive}开发了一个自适应MPM框架,用于子网格边界处理和改进的碰撞处理。MPM的一大显著优势是能够有效处理断裂现象,使其在涉及材料破裂和失效的仿真中尤为有用。Wolpher等~\cite{wolper2019cd}提出了连续损伤材料点方法,用于模拟包含大范围弹塑性变形的动态断裂。

MPM天生支持多种材料的耦合,并在网格上处理碰撞。然而,这也导致了数值粘性的相关问题。由于粒子在相同的B样条核宽度内难以相互分离,传统MPM中的流固界面通常具有无滑移边界条件。Han等~\cite{han2019hybrid} 将碰撞求解分为两个方面:超弹性能量模型,通过对碰撞施加惩罚,以及粒子-网格-粒子转移方案,类似于平滑核。许多研究者已经为解决这个问题做出了贡献。为不同材料采用不同网格是解决这一问题的常见方法。在追踪流固界面后,计算基于固相与两相之间网格速度差异的碰撞力,用于防止两相之间的粒子碰撞~\cite{yan2018mpm,han2019hybrid}。然而,这种方法仅限于显式时间积分,并且存在不稳定性问题。Fang等~\cite{fang2020iq} 提出了一种新颖的方案,使用两种不同的背景网格来模拟弹性固体与流体之间强耦合的相互作用,该方案采用了界面积分(IQ)离散化,用于实现自由滑移边界条件。在MPM中,也有采用单一网格的解决方案。Hu等~\cite{hu2018moving} 提出了CPIC转移方案,通过强制速度不连续性,支持更好的软体薄刚体耦合和流体薄刚体耦合的模拟。然而,该方法仅支持具有显式几何表示的刚体,无法模拟流体与弹性固体之间的相互作用,同时强制施加自由滑移条件。Fei等~\cite{fei2021revisiting} 基于NFLIP思想使用改进的转移方案来解锁MPM中的粒子。与传统的MPM积分器相比,该方案减少了扩散和非物理粘性,但如果位置调整参数选择不当,可能会导致不稳定性并容易出现粒子相交问题。

\subsection{自适应网格算法}
TODO

\section{研究目标和内容}
针对前文提到的问题与挑战,本文的研究目标将聚焦于设计一种全新的物质点法粒子网格传输算法,并结合自适应网格算法,以模拟流固耦合现象中流固接触与分离细节,并节省计算开销,同时设计与开发一款基于节点系统的物理仿真软件。结合国内外相关研究的成果,本文提出了一系列改进算法,其主要的研究贡献总结如下:

\begin{enumerate} 
    \item[1.] 针对传统MPM方法在统一背景网格上数值粘性的限制,本文通过修改传统的MPM流程,引入分解兼容仿射粒子网格(DC-APIC)积分器,在切向方向上强制实施自由滑移边界条件,在法向方向上实现分离条件,且通过多种复杂的流体与弹性固体相互作用的仿真进一步验证了该方法的有效性。为了寻找流固界面以及粒子与网格节点之间的兼容性,本文设计了一种新颖的基于相场梯度的分割方法,具有高精度和较高的计算效率。
    \item[2.] 针对传统MPM处理物体碰撞时会产生白色缝隙的问题。本文提出一种融合了分解兼容仿射粒子网格(DC-APIC)积分器的自适应网格算法,可以在不带来大量额外开销都同时有效保留不同尺度的碰撞细节。
    \item[3.] 设计并实现基于节点系统的物理仿真软件,其中提供了一套程序化编辑和运算的系统,支持模型导入、物理仿真、光追渲染等功能,并且在该软件框架内部实现了本文的物质点法改进算法。
 \end{enumerate}
 围绕这些研究内容,本文设置了多个具有挑战性的实验场景,例如海啸摧毁城市、不同密度橡胶玩具落水、溃坝和软体交互等,验证本文在产生真实感视觉效果方面的鲁棒性和实用性,同时还设置多个验证对比实验进行具体分析。总而言之,本文成功的提供了一个能够正确模拟流体与固体双向耦合交界面正确分离的仿真方法以及一套节点化物理仿真软件。

\section{论文组织结构}
TODO
% 论文主体部分,其他章节
\chapter{相关概念及技术} \label{chap:background}
本章将介绍物质点法模拟流固耦合的相关概念与技术,主要包括连续介质力学的弹-塑性本构模型、物质点法和自适应网格方法。本章首先介绍控制连续介质的物理方程和材料本构模型理论,然后重点介绍物质点法的数值求解流程以及不同粒子网格传输算法的优缺点,最后介绍自适应网格算法在物理模拟研究中的应用。

\section{本构模型}\label{sect:background:constitutive_model}
物质点法的物理理论来源于连续介质力学,在连续介质力学中,材料的物理特性由应力-应变关系决定,应力-应变关系就是本构关系。应变表示物体发生形变的程度,可以有被挤压和被拉伸两种状态,一般由形变梯度$\mathbf{F}$来表示。具体而言,在未形变的空间中,材料粒子的位置为$\mathbf{X}$,被映射到形变后的空间中,位置为$\mathbf{x}$,即$\mathbf{x} = \phi(\mathbf{X})$,其中$\phi$是形变映射。形变梯度$\mathbf{F}$是$\phi$的雅可比矩阵,表示为$\mathbf{F}=\frac{\partial \phi}{\partial \mathbf{X}}(\mathbf{X},t)$。应力-应变(本构关系)可以表示为:
\begin{equation}
    \mathbf{\sigma}=\frac{1}{J}\frac{\partial \mathbf{\psi}}{\partial \mathbf{F}}, \; 
    J = det(F),
\end{equation}
其中$J$是$\mathbf{F}$的行列式,表示材料粒子的体积变化,$\mathbf{\sigma}$是Cauchy应力张量,$\psi$是能量密度函数,表示在一定应变下单位材料持有的势能。通过选择不同的能量密度函数$\psi$就可以构建材料的本构模型,进而模拟出诸如水、雪、沙子、熔岩等自然环境中各不相同的物质。

物质点法的控制方程由质量守恒和动量守恒组成:
\begin{equation}
\frac{D\rho}{Dt}+\rho \nabla \cdot \mathbf{v} = 0, \;\;
\rho \frac{D \mathbf{v}}{Dt} = \nabla \cdot\mathbf{\sigma} + \mathbf{\rho} \mathbf{f}_{ext},
\end{equation}
其中$\rho$是密度,$\mathbf{v}$是速度,$\mathbf{f}_{ext}$是外力,如重力。形变梯度可以分解为$\mathbf{F}=\mathbf{F^E}\mathbf{F^P}$,其中$\mathbf{F^E}$和$\mathbf{F^P}$分别表示弹性部分和塑性部分。本文中,我们使用Neo-Hookean本构模型~\cite{DBLP:journals/tog/TuLZLWWQ24},其中,能量密度$\psi$被分解为体积部分$\psi_{vol}$和偏差部分$\psi_{dev}$:
\begin{equation}
    \begin{aligned}
        \psi_{dev} &= \frac{\mu}{2}(J^{-2/d} tr(\mathbf{F}\mathbf{F}^T) - d), \\
        \psi_{vol} &= \frac{\kappa}{2}(
        \frac{J^2-1}{2}-log(J)
        ),
    \end{aligned}
\end{equation}
其中$d$是维度,$\mu$是剪切模量,$\kappa$是体积模量,然后应力可以表示为:
\begin{equation}
\begin{aligned}
    \mathbf{\tau} &= \frac{1}{J}\mathbf{\sigma},\\
    \mathbf{\tau} &= \mathbf{\tau}_{vol}+\mathbf{\tau}_{dev},\\
    \mathbf{\tau}_{vol} &= \frac{\kappa}{2}(J^2-1)\mathbf{I}+\mu dev(J^{-2/d} \mathbf{F}\mathbf{F}^T),
\end{aligned}
\end{equation}
其中$\tau$是Kirchhoff应力,$dev(\mathbf{A})=\mathbf{A}-\frac{1}{d}tr(\mathbf{A})\mathbf{I}$表示任意张量$\mathbf{A}$的偏差部分。

大多数材料在剪切应力超过特定屈服条件时会表现出塑性变形,可以通过屈服函数$y$来描述。任何违反此条件的应力(即$y > 0$)应通过回归映射过程投影到屈服面上。本文中,我们采用了Drucker-Prager模型~\cite{klar2016drucker}来描述颗粒材料和流体。为避免体积异常增加,我们通过追踪和恢复体积变化,使用改进的回归映射算法~\cite{10.1145/3072959.3073651}。


\section{物质点法}
\subsection{求解步骤}
TODO

\subsection{粒子网格传输算法}
在MPM中,传输算法是指在每个仿真步骤中,将物理量在材料点和欧拉网格间相互传递的过程。由于MPM结合了拉格朗日和欧拉框架的优点,因此传输算法的设计对准确建模材料行为至关重要。

本文用上标$n$、$n+1$表示时间$t^n$和$t^{n+1}$时的物理量,为了简便,本文设定$\Delta t=t^{n+1}-t^n$,$\Delta \mathbf{x}_{ip}^{n}=\mathbf{x}_i-\mathbf{x}_p^n$,并且$w_{ip}=N_i(\mathbf{x}_p)$,其中$N_i(\mathbf{x}_p)$表示节点$i$和粒子$p$之间的二次B样条插值函数,主要用于传输步骤。

标准的PIC方案根据以下公式将质量$m_p$、位置$\mathbf{x}_p$和速度$\mathbf{v}_p$从粒子传输到欧拉网格:
\begin{equation}
    \begin{aligned}
    P2G: \;\;    
    &m_i\mathbf{v}_i^n =\sum_{p}w_{ip}m_p\mathbf{v}_p^n, \\
    G2P:\;\; & \mathbf{v}_p^{n+1}=\sum_iw_{ip}\mathbf{v}_i^*,\\
    &\mathbf{x}_p^{n+1}=\mathbf{x}_p^n+\Delta t\sum_iw_{ip}\mathbf{v^*}
    \end{aligned}
\end{equation}
其中$\mathbf{v}_i^*$表示在应力和碰撞解决后更新的网格速度。PIC传输方案稳定,但存在过度的耗散问题。

FLIP传输方案\cite{zhu2005animating}通过添加一个额外的粒子速度保持项来减轻PIC中的耗散,表达为:
\begin{equation}
    \begin{aligned}
    P2G: \;\;    
    &m_i\mathbf{v}_i^n =\sum_{p}w_{ip}m_p\mathbf{v}_p^n, \\
    G2P:\;\; & \mathbf{v}_p^{n+1}=\sum_iw_{ip}\mathbf{v}_i^* + \alpha(\mathbf{v}_p^n-\sum_iw_{ip}\mathbf{v}_i^n),\\
    &\mathbf{x}_p^{n+1}=\mathbf{x}_p^n+\Delta t\sum_iw_{ip}\mathbf{v^*},
    \end{aligned}
    \label{eq:flip}
\end{equation}
其中$\alpha$是PIC-FLIP混合比率~\cite{bridson2015fluid}。当$\alpha=0$时,公式~\eqref{eq:flip}回归为PIC,当$\alpha=1$时,它是完全的FLIP。尽管FLIP传输方案通过额外的粒子速度保持项使仿真更加生动,但它仍然存在不稳定性。

\begin{figure}
    \centering
    \includegraphics[width=1\linewidth]{figures/asflip.png}
    \caption{APIC、ASFLIP和DC-APIC二维流固分离测试}
    \label{fig:asflip}
\end{figure}

APIC \cite{jiang2015affine}方案通过保持角动量和防止不稳定性来减少一部分耗散。APIC中的P2G和G2P传输步骤如下:
\begin{equation}
    \begin{aligned}
    P2G: \;\;    
    &m_i\mathbf{v}_i^n =\sum_{p}w_{ip}m_p(\mathbf{v}_p^n + \mathbf{B}_p^n(\mathbf{D}_p^n)^{-1}\Delta\mathbf{x}_{ip}), \\
    G2P:\;\; & \mathbf{v}_p^{n+1}=\sum_iw_{ip}\mathbf{v}_i^*,\\
    &\mathbf{B}_p^{n+1} = \sum_{i}w_{ip}\mathbf{v}_i^*\Delta\mathbf{x}_{ip},\\
    &\mathbf{x}_p^{n+1}=\mathbf{x}_p^n+\Delta t\sum_iw_{ip}\mathbf{v^*},
    \end{aligned}
\end{equation}
其中$\mathbf{D}_p^n$类似于惯性张量,表示为
\begin{equation}
    \mathbf{D}_p^n=\sum_{i}w_{ip}\Delta\mathbf{x}_{ip}(\Delta\mathbf{x}_{ip})^T
\end{equation}
局部速度仿射项$\mathbf{B}_p\mathbf{D}_p^{-1}$有助于保持线性惯量和转动惯量,并且这一过程保证了动量守恒。然而,APIC仍然基于材料连续性的假设,因此无法有效处理粒子分离的情况。

Affine-augmented Separable FLIP(ASFLIP)方案\cite{fei2021revisiting}旨在解决材料连续性假设可能不成立的情况,例如对于高度分散的沙子。它在FLIP中添加了粒子速度保持项,并在APIC中添加了一个额外的位置调整项,以便更容易地实现粒子分离:
\begin{equation}
    \begin{aligned}
    P2G: \;\;    
    &m_i\mathbf{v}_i^n =\sum_{p}w_{ip}m_p(\mathbf{v}_p^n + \mathbf{B}_p^n(\mathbf{D}_p^n)^{-1}\Delta\mathbf{x}_{ip}), \\
    G2P:\;\; & \mathbf{v}_p^{n+1}=\sum_iw_{ip}\mathbf{v}_i^*+ \alpha(\mathbf{v}_p^n-\sum_iw_{ip}\mathbf{v}_i^n),\\
    &\mathbf{B}_p^{n+1} = \sum_{i}w_{ip}\mathbf{v}_i^*\Delta\mathbf{x}_{ip},\\
    &\mathbf{x}_p^{n+1}=\mathbf{x}_p^n+\Delta t(\sum_iw_{ip}\mathbf{v^*}+ \beta_p\alpha(\mathbf{v}_p^n-\sum_iw_{ip}\mathbf{v}_i^n)),
    \end{aligned}
\end{equation}
其中$\beta_p$是控制位置调整项的影响比例。然而,它仍然存在数值粘性问题,在流固耦合问题中尤为突出,当固体和流体都被离散化为MPM粒子,这两个相位交接处会难以平滑分离。此外,ASFLIP的效果受到参数的影响,这使得它更容易产生材料穿透现象,如图~\ref{fig:asflip}所示。

总结来说,对于混合粒子-网格方法,数值粘性来自粒子和网格之间的传输。通常,粒子的数量远大于网格节点的数量。因此,P2G步骤可能导致信息丢失。如图~\ref{fig:p2g}(a)所示,当两个质量相等且速度相反的粒子将动量传输到网格时,速度将被平均化。从另一个角度来看,MPM基于材料连续性的假设,这使得它难以在物理上描述材料分离,特别是在流固耦合的情况下。两个相位之间的界面速度场很容易被P2G步骤平滑化。

\begin{figure}[h]
    \centering
    \includegraphics[width=0.75\linewidth]{figures/p2g.png}
    \caption{二维情况下APIC和DC-APIC对粒子的对流效果对比}
    \label{fig:p2g}
\end{figure}

\chapter{分解兼容仿射粒子网格(DC-APIC)积分器} \label{chap:dcapic}
针对物质点法在进行物理仿真时存在的难以约束自由滑动边界条件的问题,本章节提出了分解兼容仿射粒子网格(DC-APIC)积分器算法,有效地解决传统MPM算法中数值粘性问题。本章首先提出粒子与网格兼容性的定义,并给出两种计算方式:基于符号距离函数场(SDF)的方法和基于相场梯度的方法;接着分辨阐述DC-APIC粒子到网格传输算法和网格到粒子传输算法,详细给出了质量和动量的传输公式,并论述了该方法维持动量守恒的特性;然后给出结合DC-APIC的物质点法的算法全流程,着重描述了在处理交接面几何边界时采用的不同法线计算方式;最后在多个二维和三维的实验中对比了本文同其他传统传输方式的优点,并罗列了时间开销。

\section{粒子与网格的兼容性}
本文采用了来自兼容性粒子网格方法(Compatible Particle in Cell, CPIC)~\cite{hu2018moving}中的兼容性概念来标记粒子与网格节点之间的关系。在CPIC中,当粒子$p$和节点$i$位于同一不被刚性碰撞物体分隔的区域内时,它们是兼容的;反之则为不兼容的。由于多个薄刚性物体可以将一个弹性物体分割成两个或更多区域,因此与同一网格节点兼容的两个粒子可能属于不同的区域。

在本文中,我们为每个粒子分配了一个附加属性$s$,表示从固体($s$ = 1)到流体($s$ = 0)的材料状态。我们定义粒子$p$与节点$i$为兼容状态当且仅当它们相位相同,即粒子$p$和节点$i$间的兼容性$c_{ip}$当前仅当$s_p = s_i$时为1。由于只存在固体和流体两个相位,与同一网格节点兼容的两个粒子必须属于相同区域。因此,不需要使用~\cite{hu2018moving}中的颜色标记距离向量场(Colored Distance Field, CDF)数据结构来计算兼容性。

\begin{figure}[htbp]
    \centering
    \includegraphics[width=0.6\linewidth]{figures/phase-field-gradient.png}
    \caption{基于相场梯度的粒子与网格兼容性计算方法流程}
    \label{fig:phase-field-gradient}
\end{figure}

根据粒子位置,使用各向异性核~\cite{yu2013reconstructing}可以平滑地重建出符号距离函数场(Signed Distance Field, SDF),进而比较网格格点的符号距离函数值来计算网格节点的相位$s_i$,但是其时间复杂度非常高,在场景测试中几乎占用了每个时间步长的一半时间。因此,我们设计了一种基于相场梯度的方法,该方法最早用于断裂仿真~\cite{homel2017field}。首先,通过粒子相位状态估计网格相位梯度,计算公式为$\nabla s_i = \sum_{p} s_p \nabla w_{ip}$。其次,使用$\mathbf{Q}_p$表示材料加权的粒子相位梯度,$\mathbf{Q}_p = (-1)^{s_p}\sum_{i} \nabla s_i w_{ip}$。然后,根据粒子相位和网格节点的相位梯度的点积符号将网格节点划分为不同的相位,公式为
\begin{equation}
s_{i}=
\begin{cases} 
s_p, &  \textbf{sgn}(\nabla s_i \cdot \mathbf{Q}_p) > 0 \\
1 - s_p, & \text{otherwise}.
\end{cases}
\end{equation}
然而,两个不同相位的粒子可能会将一个网格节点划分为它们各自的相位,这意味着$\exists p, q, i,$ 使得$\textbf{sgn}(\nabla s_i \cdot \mathbf{Q}_p) > 0 \land \textbf{sgn}(\nabla s_i \cdot \mathbf{Q}_p) > 0$,其中$s_p$与$s_q$不同。这种病态情况在固体区域的锐利边界处可能频繁发生。我们采用启发式算法来解决这个问题(如图~\ref{fig:phase-field-gradient}所示)。在边界区域,粒子相位梯度在网格相位梯度方向上的投影长度$\| \nabla s_i \cdot \mathbf{Q}_p \|$越大,对网格节点相位的影响越显著。因此,我们使用梯度的点积作为确定网格节点相位的权重。我们定义$I_p=\sum_{p}(\nabla s_i \cdot \mathbf{Q}_p)$。如果$I_p > 0$,我们将粒子$p$分类为固体粒子,并赋值$s_p=1$。反之,如果$I_p<0$,我们将粒子$p$分类为流体粒子,并赋值$s_p=0$。具体流程如图~\ref{fig:phase-field-gradient}所示,二次B样条核范围由紫色方框标注。图中(a)展示了背景网格与B样条核的范围,其中计算的网格节点位于中心。蓝色箭头是网格节点相位值的梯度$\nabla s_i$;(b) 中粒子颜色表示固体-流体相位。蓝色表示流体粒子,红色表示固体粒子。红色箭头是材料加权的粒子相位梯度值$\mathbf{Q}_p$。(c)中标明了粒子与背景网格节点。绿色箭头表示从附近粒子计算的网格节点相位贡献估计值,通过$\nabla s_i\cdot \mathbf{Q}_p$得到,通过它们所获得的总贡献的信号计算网格节点相位。(d)为最后一步,计算所有网格节点的相位值。



%-------------------------------------------------------------------------

\section{粒子-网格传输算法}
\subsection{DC-APIC粒子到网格传输算法}
\label{sec:DC-APIC_p2g}
本章使用下标$p$、$q$表示粒子,下标$i$、$j$表示网格。$i^{p+}$表示与粒子$p$兼容的节点,$i^{p-}$表示不兼容的节点。同样,$p^{i+}$表示与网格节点$i$兼容的粒子,$p^{i-}$表示不兼容的粒子。本文方法中,在大多数仿真区域内部仍然使用传统APIC传输方案,而在接口处,粒子仅将动量的法向分量传递到不兼容的网格节点,以避免在另一个相位中平滑速度,同时粒子将动量的法向分量和切向分量都传递到兼容的网格节点,以支持流固耦合过程中自动的MPM碰撞求解,公式如下:
\begin{equation}
\begin{aligned}
    m_i &= \sum_{q\in\{p^{i+},p^{i-}\}}w_{iq}m_q, \\
    m_{i+} &= \sum_{q\in\{p^{i+}\}}w_{iq}m_q, \\
    (m_i\mathbf{v}_i)^{norm} &=\sum_{q\in\{p^{i+},p^{i-}\}}w_{ip}m_q(\mathbf{v}_q\mathbf{n}_q + \mathbf{B}_q\mathbf{D}_q^{-1}\Delta\mathbf{x}_{iq}), \\
    (\mathbf{m}_i\mathbf{v}_i)^{tan} &=\sum_{q\in\{p^{i+}\}}w_{iq}m_q(\mathbf{v}_q - \mathbf{v}_q\mathbf{n}_q), \\
    \mathbf{v}_i &=(m_i\mathbf{v}_i)^{norm}/m_i+(m_i\mathbf{v}_i)^{tan}/m_{i+},
\end{aligned}
\end{equation}
其中$m_i$是通过映射粒子传输到节点的总质量,$m_{i+}$仅是通过兼容粒子传输的质量,本地速度仿射项$\mathbf{B}_q\mathbf{D}_q^{-1}$有助于保持能量,否则会因如PIC等传输方案中的数值粘性而损失。我们将动量分为法向分量和切向分量,因为它们需要在速度计算步骤中除以不同的质量。具体来说,如果切向速度是$(m_i\mathbf{v}_i)^{tan}/m_i$,那么速度将被耗散。

如图~\ref{fig:p2g}所示,APIC方案在同一网格节点内分离朝相反方向运动的粒子时相对困难(在BSpline核范围内),因为粒子到网格操作会对核范围内的粒子动量进行平均,从而导致数值粘性。相反,DC-APIC将动量分解为切向分量和法向分量,仅将切向动量传递到兼容的网格节点。这避免了来自另一个相位的粒子的平均影响。此外,由于法向速度仍然按正常方式传输,并且没有直接将位置调整项添加到$\mathbf{x}_p$,它有效地解决了NFLIP~\cite{stomakhin2013material}中发现的粒子穿透问题。

\subsection{DC-APIC网格到粒子传输算法} \label{sec:DC-APIC_g2p}
对于每个粒子,由于不连续性约束的施加,不兼容网格节点上的切向速度与粒子无关。我们采用了幽灵粒子速度方法,在这种方法中,我们假设对于任意节点 $j\in i^{p-}$,其速度通过从粒子$p$的上一时间步速度常数外推得到,即 $v_j=v_p^n$。因此,从网格到粒子的 DC-APIC传输方法,综合了来自兼容节点和不兼容节点的贡献,其表达式为:
\begin{equation}
    \label{eq:DCAPIC_g2p}
    \begin{aligned}
        \mathbf{v}_p &=\sum_{j\in i^{p+}}w_{jp}\mathbf{v}_j^* + \sum_{j\in i^{p-}}w_{jp}\mathbf{v}_j^*\mathbf{n}_j + \sum_{j\in i^{p-}}w_{jp}(\mathbf{v}_p-\mathbf{v}_p\mathbf{n}_j), \\
        \mathbf{B}_p &= \sum_{j\in i^{p+}}w_{jp}\mathbf{v}_j^*\mathbf{x}_{jp} + \sum_{j\in i^{p-}}w_{jp}(\mathbf{v}_p-\mathbf{v}_p\mathbf{n}_j)\Delta\mathbf{x}_{ip}.
    \end{aligned}
\end{equation}

粒子将从兼容节点获得速度,并且从不兼容节点获得法向速度分量,以实现自动MPM耦合而不发生自穿插。我们只在公式~\eqref{eq:DCAPIC_g2p}中使用幽灵粒子速度的切向分量,以确保稳定性。需要注意的是,这里使用的法向量必须是网格节点法向量 $\mathbf{n}_i$,以保证动量守恒。动量守恒可以在APIC积分器中严格证明,其中在G2P和P2G过程前后的总粒子动量等于总网格动量。然而,DC-APIC算法使用了CPIC技术,其中P2G根据流固界面的几何形状选择性地将部分动量传递到网格,并且在G2P过程中,通过使用 $\mathbf{v}_p^n$ 作为近似值,将失去的动量传回粒子。从另一个角度来看,本文的传输方案可以视为两种操作的组合:正交分解和 CPIC。显然,正交分解保持了总动量;在CPIC的P2G过程中,速度被传递到兼容节点。然后,在G2P过程中,不兼容节点将幽灵速度传回粒子。这些幽灵速度与前一时刻的粒子速度相同。因此,整个P2G-G2P流程看起来像是在APIC积分器与真实数据之间进行插值,其中真实数据指的是速度与前一时刻保持一致,从而确保动量守恒。

\section{基于DC-APIC传输算法的MPM工作流程}
\begin{figure*}[htbp]
\centering
\includegraphics[width=0.9\linewidth]{figures/pipeline.png}
\caption{基于DC-APIC积分器的MPM算法流程} 
\label{fig:pipeline}
\end{figure*}

与传统的MPM工作流程相比,本文增加了一个额外的步骤来获取粒子与网格的兼容性信息,以调节固流耦合中网格到粒子和粒子到网格的传输过程,处理材料间的不连续性。图~\ref{fig:pipeline} 展示了我们数值求解器的主要数据流,其主要工作流程如下所列:

\begin{enumerate}
\item[(1)] \textbf{界面信息计算。} 计算粒子法向量 $n_p$ 和网格节点法向量 $n_i$ 以及网格上的流固相场状态 $s_i$,这些信息用于控制 DC-APIC 的粒子到网格和网格到粒子步骤。
\item[(2)] \textbf{粒子到网格传输。} 根据 DC-APIC方案将质量和动量从粒子传递到网格,具体细节参见第~\ref{sec:DC-APIC_p2g}节。
\item[(3)] \textbf{网格速度更新。}
根据邻近粒子的变形梯度 $\mathbf{F}_p^n$ 计算施加于节点 $i$ 上的力 $\mathbf{f}_i^n=-\sum_p V_p^0 \frac{\partial \Psi}{\partial F} {\mathbf{F}_p^n}^T \nabla w_{ip}^n$,然后以显式欧拉方式更新速度:$\mathbf{v}_i^{n+1}=\mathbf{v}_i^n + \Delta t \mathbf{f}_i^n / m_i$。
\item[(4)] \textbf{网格到粒子传输。}
根据第~\ref{sec:DC-APIC_g2p}节的 APIC 方法更新速度 $\mathbf{v_p^{n+1}}$、粒子的空间速度梯度 $(\nabla \mathbf{v})_p^{n+1}$ 和仿射状态 $\mathbf{B}_p^{n+1}$,然后进行粒子推进与碰撞处理。
\item[(5)] \textbf{应变更新。}
更新粒子的试验形变梯度:$\mathbf{F}_p^{n+1} = (I + \Delta t \nabla \mathbf{v}_p^{n+1}) \mathbf{F}_p^n$。

\end{enumerate}



\subsection{界面信息计算}
\label{interface_information_calculation}
在 \textbf{IIC} 步骤的开始,我们使用额外的 P2G 步骤仅传递相场。接收到流体和固体相的网格节点被标记为边界节点,映射到这些节点的粒子被标记为边界粒子。

\subsection{法向量估计}
传统的有符号距离函数(SDF)方便执行内外查询和法向量估计。然而,每次时间步重建粒子水平集是非常耗时的。我们采用与~\cite{fang2020iq} 相同的方法,通过选择固体粒子的负质量梯度场来确定固体的法向量。
\begin{equation}
\begin{aligned}
m_i^s &= \sum_{p^s} m_p w_{ip},\\
\mathbf{n}_p^s &= -\sum_i m_i^s \nabla w_{ip} / \| \sum_i m_i^s \nabla w_{ip} \| ,\\
\end{aligned}
\end{equation}
其中 $m_i^s$ 是映射到一个网格节点的固体质量,$p^s$ 表示固体粒子,$\mathbf{n}_p^s$ 表示固体粒子的法向量。对于网格法向量和流体粒子的法向量,我们设计了一种不同的策略,如下所示:
\begin{equation}
\label{eq:ni_npf}
\begin{aligned}
\mathbf{n}_i &= \sum_{p^s} \mathbf{n}_p^s w_{ip} / \| \sum_{p^s} \mathbf{n}_p^s w_{ip} \| ,\\
\mathbf{n}_p^f &= -\sum_i \mathbf{n}_i w_{ip} / \| \sum_i \mathbf{n}_i w_{ip} \|,\\
\end{aligned}
\end{equation}
其中 $\mathbf{n}_i$ 是网格节点的法向量,可以覆盖所有界面网格节点,$\mathbf{n}_p^f$ 表示流体粒子的法向量。从 Eq.~\eqref{eq:ni_npf} 可以看出,我们首先使用 B 样条平滑核对固体粒子法向量 $\mathbf{n}_p^s$ 进行平滑,以得到网格节点的法向量。平滑核可以使 DC-APIC 的网格到粒子(第~\ref{sec:DC-APIC_p2g} 节)和粒子到网格(第~\ref{sec:DC-APIC_g2p} 节)更加稳定。然后,我们利用网格节点法向量来计算流体粒子的法向量。该策略既稳定又高效,避免了在计算流体法向量时需要通过空间数据结构查找最近的固体粒子。


\section{实验结果与分析}
通过将本章节提出的 DC-APIC积分器结合MPM算法流程,实现CPU端仿真程序,并进行对比实验来验证该算法的有效性。具体的实验场景参数设置见表\ref{tab:demo_parameter}。
\begin{table*}[h]
    \centering
    \caption{\textbf{Parameters and performance.}}
    \resizebox{\textwidth}{!}{%
    \begin{tabular}{l@{\hspace{1cm}}|c|c|c|c|c|c|c}
    \hline
    Example & 
    sec/frame & 
    $\Delta x$ &
    $\Delta t$ &
    \#Particles &
    Young's Modulus &
    Poisson's Ratio &
    density ratio
    \\
    \hline
    \makebox[7em][l]{(Fig.~\ref{fig:bunny3d} left)} Pinned bunny & 6.33  & 0.1 & 0.001 & 98K & 8000 & 0.4  & 1.0\\
    \makebox[7em][l]{(Fig.~\ref{fig:bunny3d} right)} Pinned bunny & 3.60  & 0.1 & 0.001 & 98K & 8000 & 0.4  & 1.0\\
    \makebox[7em][l]{(Fig.~\ref{fig:bunny_bath})} Bear bath & 24.17 & 0.1 & $5 \times 10^{-4}$ & 400K & 6000 & 0.4 & 1.2 \\
    \makebox[7em][l]{(Fig.~\ref{fig:duck})} Duck & 57.63 & 0.1 & 0.001 & 640K & 6000 & 0.4 & 0.5/1.0/2.5 \\
    
    \makebox[7em][l]{(Fig.~\ref{fig:jelly} left)} Jelly Sand & 14.72 & 0.05 & $5 \times 10^{-4}$ & 1.2M & 1500 & 0.2 & 1.0 \\
    \makebox[7em][l]{(Fig.~\ref{fig:jelly} middle)} Jelly Sand & 17.16 & 0.05 & $5 \times 10^{-4}$ & 1.2M & 4000 & 0.3 & 1.0 \\
    \makebox[7em][l]{(Fig.~\ref{fig:jelly} right)} Jelly Sand & 15.91 & 0.05 & $5 \times 10^{-4}$ & 1.2M & 6000 & 0.4 & 1.0 \\
    \makebox[7em][l]{(Fig.~\ref{fig:sdf_phase_compare} top)} 
     Balls(2D) & 0.78 & 0.05 & 0.001 & 8K & 4000 & 0.4 & 1.0 \\
    \makebox[7em][l]{(Fig.~\ref{fig:sdf_phase_compare} middle)} 
     Balls(2D) & 0.42 & 0.05 & 0.001 & 8K & 4000 & 0.4 & 1.0 \\
    \makebox[7em][l]{(Fig.~\ref{fig:sdf_phase_compare} bottom)} 
     Balls(2D) & 1.14 & 0.05 & 0.001 & 8K & 4000 & 0.4 & 1.0 \\
    \hline
    \end{tabular}%
    }
    
    \label{tab:demo_parameter}
\end{table*}

\begin{figure}[htbp]
    \centering
    \includegraphics[width=0.7\linewidth]{figures/sdf_phase_compare.png}
    \caption{二维水球坠落实验:两种计算兼容性方法的DC-APIC算法和APIC算法的对比}
    \label{fig:sdf_phase_compare}
\end{figure}

图~\ref{fig:sdf_phase_compare}中通过二维水块落在弹性软体球上的实验,比较了基于粒子重建SDF和基于相场梯度这两种方法计算粒子网格兼容性的效果对比。使用本文提出的基于相场梯度计算兼容性的方法可以取得SDF方法类似的效果,并且开销更小。图中还额外对比了APIC积分器的效果,其模拟的耦合存在不真实的粘度,水很难离开弹性体;而另外两种使用DC-APIC积分器的方法都使得水能够自由滑落球面。


\begin{figure}[H]
    \centering 
    \includegraphics[width=1\linewidth]{figures/bunny3d.png} 
    \caption{三维沙堆砸落碰撞弹性兔子并分离实验} 
    \label{fig:bunny3d} 
\end{figure}

图~\ref{fig:bunny3d}展示了三维场景下,本文的DC-APIC方法能够有效解决具有自由滑移条件的流固耦合边界条件,而传统的基于DC-APIC积分器的MPM方法难以分离沙粒流与弹性固体兔子。

\begin{figure}[H]
    \centering
    \includegraphics[width=0.8\linewidth]{figures/duck.png}
    \caption{流固耦合固体密度测试}
    \label{fig:duck}
\end{figure}
图~\ref{fig:duck}通过不同固体与流体密度比,测试了本文DC-APIC方法能够正确处理浮力,实现物理正确的固体落水的漂浮、悬浮或沉底效果。

\begin{figure}[H]
    \centering
    \includegraphics[width=0.7\linewidth]{figures/jelly_6.png}
    \caption{三维不同材料果冻同颗粒流交互实验}
    \label{fig:jelly}
\end{figure}
图~\ref{fig:jelly}展示了果冻材料的固体同无粘性颗粒流的交互动画,不同颜色的果冻有不同的物理参数,表现出不一样的最大形变量和弹性系数,颗粒流在最后均可以顺利同果冻分离而不发生粘滞现象。

\begin{figure}[H]
    \centering
    \includegraphics[width=0.7\linewidth]{figures/bunny_bath.png}
    \caption{溃坝冲击三维实验}
    \label{fig:bunny_bath}
\end{figure}
在图~\ref{fig:bunny_bath}中,水体溃坝并击倒一只由超弹性材料构成的小熊玩具并将其冲走,模拟实验中水粒子正确地绕过弹性表面流动,没有产生非物理的粘性。

\begin{figure}[H]
    \centering
    \includegraphics[width=1\linewidth]{figures/echarts.png}
    \caption{DC-APIC算法、APIC算法和IQ-MPM算法的时间复杂度对比}
    \label{fig:echarts}
\end{figure}
在上文的众多实验中,流体与固体相互作用发生前,DC-APIC算法、APIC算法和IQ-MPM算法的平均计算效率非常相似。然而,在相互作用发生后,IQ-MPM需要大量时间计算流体-固体边界上的压力,这导致其处理时间明显长于传统MPM。相比之下,DC-APIC使用统一的背景网格,不需要额外的求解器,因此其计算速度与传统MPM相当。三种传输方案的平均时间消耗(秒/帧)对比如图~\ref{fig:echarts}所示。

\begin{figure}[H]
    \centering
    \includegraphics[width=0.8\linewidth]{figures/energy.png}
    \caption{粒子动能变化曲线实验}
    \label{fig:energy}
\end{figure}
图~\ref{fig:energy}中展示了模拟过程中,粒子的总动能变化曲线,在三维鸭子落水的实验中,流体撞击固体后,APIC方法中的动能比DC-APIC方法中的动能更快地耗散。


\section{本章小节}
本章介绍了本文的核心创新点:分解兼容仿射粒子网格(DC-APIC)积分器。该积分器通过在兼容条件和非兼容条件下传递不同的动量,既能实现流固边界的顺利分离,又保留了传统MPM方法可以自动处理碰撞、不发生物体穿透现象的优点。本章节提供的结合DC-APIC的MPM方法,能够正确处理流固界面演变、固体受浮力影响动画,在二维和三维的多个实验中展现出鲁棒性和优秀的性能表现,结合后文的自适应网格算法,本文将进一步增加流固耦合交界面的细节真实性并优化性能表现。
\chapter{基于DC-APIC积分器的自适应广义插值物质点法}
物体接触、自碰撞、边界处理问题在物质点法中由粒子网格系统自动处理,虽然减少了程序设计上的难度,也引入了新的问题。每当两个粒子影响到某些共同的网格格点时,物质点法就将这两个粒子视为处于接触状态。因此,分离距离本质上与网格单元间距$\Delta x$成正比。即使是对于动力学非常简单的模拟,为了避免出现视觉上明显可见的碰撞间隙,也需要采用高分辨率网格。此外,物质点法无法处理比网格分辨率更精细的边界条件。这意味着材料无法穿过小于$\Delta x$的孔洞,同样地,厚度小于$\Delta x$的刀片也无法切割材料。本章将提供一个结合DC-APIC积分器的广义插值物质点法,同时支持自适应网格划分,旨在进一步提高DC-APIC解决物质点法数值粘性的效果与计算效率,同时克服物质点法固有的碰撞边界细节处理问题。本章首先完整介绍自适应广义插值物质点算法,包括广义插值物质点法、自适应基函数、多层网格间的嵌入与插值机制、网格细分和粒子重采样;然后结合分解兼容仿射粒子网格(DC-APIC)积分器,给出基于DC-APIC积分器的自适应广义插值物质点法的算法流程,包括法系计算、粒子兼容性计算和粒子网格传输。本章节的算法经过验证可以更好的模拟细粒度材料自碰撞边界形态以及流固分离现象,同时自适应算法拥有更好的计算效率。

\section{自适应广义插值物质点法}

\subsection{广义插值物质点法}
传统MPM算法中将粒子$p$和格点$i$间的插值权重$w_{ip}$被定义为和插值函数$N_i(\mathbf{x}_p)$相等:
\begin{equation}
    w_{ip} = N_i(\mathbf{x}_p).
\end{equation}
于是权重梯度可以表示为$\nabla w_{ip} = \nabla^{\mathbf{x}}N_i(\mathbf{x}_p)$。为了提高数值稳定性,$N_i(\mathbf{x}_p)$一般采用$C^1$或$C^2$连续的插值函数,比如二次或三次B样条函数。插值函数的选取需要满足以下五条性质:
* \textbf{守恒性:} 为满足质量和动量守恒,需要保证$\sigma_iw_{ip}=1,\forall \mathbf{x}_p$。
* \textbf{非负性:}$w_{ip}\geq 0$。
* \textbf{插值性:}$\mathbf{x}_p=\sum_iw_{ip}\mathbf{x}_i$。
* \textbf{连续性:}$w_{ip}$至少满足一阶连续性。
* \textbf{局部性:}只有$\mathbf{x}_p$在$\mathbf{x}_i$附近时才不为0,距离超过核半径的都为0。

B样条函数正好符合这五条性质,但它很难拓展到自适应网格上。另一种可替代的权重梯度计算方式是广义插值物质点法(Generalized Interpolation Material Point, GIMP)~\cite{bardenhagen2004generalized}。GIMP不再限制$N_i(\mathbf{x}_p)$为$C^1$连续,而只需要$C^0$连续,比如标准的多线性插值函数,但粒子网格权重$w_{ip}$需保证$C^1$连续,其定义如下:
\begin{equation}
    \begin{aligned}
        \hat{V}_p &= \int_{\Omega\cap\Omega_p} d\mathbf{x} \\
        w_{ip} &= \frac{1}{\hat{V}_p} \int_{\Omega} \chi_p(\mathbf{x}) N_i(\mathbf{x}) \, d\mathbf{x},
    \end{aligned}
    \label{eq:GIMP-0}
\end{equation}
其中$\hat{V}_p$是粒子的体积,$\Omega$是整个仿真区域,$\Omega_p$是物质点粒子覆盖的区域,$\chi_p(x)$是粒子的特征函数,在GIMP中一般定义如下:
\begin{equation}
    \chi_p(\mathbf{x}) = 
\begin{cases}
1 & \text{if } \mathbf{x} \in \Omega_p, \\
0 & \text{otherwise}.
\end{cases}
\end{equation}
本文中把$\Omega_p$设定为以粒子位置$\mathbf{x}_p$为中心、边长为$L_p$的正方形,公式~\ref{eq:GIMP-0}可以简化为:
\begin{equation}
    \begin{aligned}
        w_{ip} &= \frac{1}{\hat{V}_p} \int_{\Omega} N_i(\mathbf{x}) \, d\mathbf{x},
    \end{aligned}
\end{equation}
权重的梯度可以类似地定义为:
\begin{equation}
    \nabla w_{ip} = \frac{1}{\hat{V}_p} \int_{\Omega_p} \nabla^{\mathbf{x}} N_i(\mathbf{x}) d\mathbf{x}。
\end{equation}

标准的多线性插值函数$N_i(\mathbf{x})$的定义如下:
\begin{equation}
    \begin{aligned}
        N_i(\mathbf{x}) &= 
                N(\frac{x_p-x_i}{\Delta x})
                N(\frac{y_p-y_i}{\Delta y})
                N(\frac{z_p-z_i}{\Delta z})\\
        N(x) &=
            \begin{cases}
            1 - |x| & 0\leq |x|<1 \\
            0 & 1\leq |x|
            \end{cases}
    \end{aligned}
\end{equation}
该公式以三维为例,其中$\mathbf{x_p}=[x_p,y_p,z_p]$是粒子位置三个分量,$\mathbf{x_i}=[x_i,y_i,z_i]$是格点位置三个分量,由于$N_i(\mathbf{x})$在每一个分量上都是线性的并且$C^0$连续,$w_{ip}$是对$N_i(\mathbf{x})$的积分操作,所以$w_{ip}$是$C^1$连续,$\nabla w_{ip}$ $C^0$连续,此外基于这样的定义,$w_{ip}$满足前文提到的五个性质。可以发现,如果$L_p=\Delta x$, GIMP计算得到的$w_{ip}$同传统MPM中的二次B样条插值函数一致,以一维情况为例:
\begin{equation}
    \begin{aligned}
        w_{ip} &= \frac{1}{\hat{V}_p} \int_{\Omega_p} N_i(t) dt\\
&= \frac{1}{\hat{V}_p} \int_{x - \frac{1}{2}}^{x + \frac{1}{2}} N_i(t) dt\\
&= 
\begin{cases}
\frac{3}{4}-|x|^2, & 0 \leq |x| <\frac{1}{2}\\
\frac{1}{2}(\frac{3}{2}-|x|)^2, & \frac{1}{2}\leq |x|<\frac{3}{2}\\
0, & \frac{3}{2}\leq |x|
\end{cases}
    \end{aligned}
\end{equation}

尽管在均匀网格上,基于B样条的物质点法和广义插值物质点法(GIMP)可以实现等效,但是将高阶B样条推广到自适应网格缺十分困难,挑战主要来自于T型连接点以及不同层级间过渡网格单元的处理,而GIMP的定义天然适合自适应网格,只需要$N_i(\mathbf{x})$满足$C^0$连续要求。

\subsection{自适应基函数}
对于自适应网格算法而言,在模拟过程中网格会在一定条件下细化分裂或泛化合并,在粗细网格交界处会形成一些复杂情况,如图~\cite{fig:freeTjunk},如何正确的构造$C^0$连续的$N_i(\mathbf{x})$对模拟结果正确性至关重要,这一节将阐述如何正确构造以应对所有复杂的情况。
\begin{figure}[H]
    \centering
    \includegraphics[width=1\linewidth]{fig/Tjunc0.png}
    \caption{自适应网格的T连接情形}
    \label{fig:freeTjunk}
\end{figure}
\subsubsection{无约束T型连接}
在粗粒度网格和细粒度网格过度的边界地带会出现T型连接,如图~\ref{fig:freeTjunk}所示,绿色的节点为T型连接节点,在~\cite{LIAN2015291}的工作中,这些节点也被看作真实节点,构成网格的自由度。这种做法在某些情况下会出现某些格点基函数为负的问题。以图~\ref{fig:freeTjunk}(b)为例。下文中,$H_\gamma^k$表示网格$\gamma$内由双线性插值构成的节点$k$的基,$N_\gamma^k$表示最终GIMP需要的网格插值函数基。为了保证$N_b$沿着a-b-c边的连续性,$N_d$沿着c-d-e边的连续性,可以得到:
\begin{equation}
    \begin{aligned}
        N_b^1 &=\frac{1}{2}(1+x)(1-|y|), \\
        N_d^1 &=\frac{1}{2}(1-|x|)(1+y), \\
        H_c^1 &= \frac{1}{4}(1 + x)( 1 - y), \\
        N_c^1 &= H_c^1 - \frac{1}{2}N_b^1 - \frac{1}{2}N_d^1.
    \end{aligned}
\end{equation}
此时不难发现$N_c^1$的值可能为负,比如当$x=0.5,y=0.1$时。在物质点法中,负权重会导致不稳定性和严重的精度损失。例如,可以构建这样一种情况:两个粒子以恰好相反的权重影响一个网格节点。此时节点质量会变为零,从而导致网格节点速度的值不正确。随后,网格节点速度的更新以及从网格到粒子的传递都会出错,进而使模拟变得不稳定或不符合物理规律。图~\ref{fig:unstable}展示了在边界处出现负数质量格点产生的穿透现象。

\begin{figure}[H]
    \centering
    \includegraphics[width=1\linewidth]{fig/unstable_case.png}
    \caption{负权重导致的不稳定现象}
    \label{fig:unstable}
\end{figure}
\subsubsection{有约束T型连接}
本文提出一种基于约束方式来得到正确的自适应网格基函数,下文将结合图~\ref{fig:free_T}所示自适应网格结构阐述该算法。图中红色的为真实节点(DOF节点),绿色的为T型连接节点,真实节点一定是某个或某几个网格的顶点,而T型连接节点会作为某个网格的边上的点(非端点)出现。


\begin{figure}[H]
    \centering
    \includegraphics[width=1\linewidth]{fig/free_T.png}
    \caption{有约束T型连接示意图}
    \label{fig:free_T}
\end{figure}

首先计算所有节点(包括真实节点和T型连接节点)的形状函数$H_{\gamma}=\sum_nH_{\gamma}^n$,表示节点$\gamma$周围所有以$\gamma$为顶点的网格(标记为$n$)的双线性插值函数基之和。比如图~\ref{fig:free_T}中,$H_a=H_a^1+H_a^2$以及$H_b=H_b^2+H_b^3$,这一步计算完成后的形状函数可视化如图~\ref{fig:shapefunc}所示。使用这些没有施加约束的形状函数作为基,可以从格点插值出一个标量场$q_(\mathbf{x})=\sum_{\gamma} q_{\gamma}H_{\gamma}(\mathbf{x})$。显然,这一步的$q_(\mathbf{x})$是不连续的,因为形状函数本身是不连续的(如图~\ref{fig:shapefunc}所示),尤其是在T型结构附近。

\begin{figure}[H]
    \centering
    \includegraphics[width=1\linewidth]{fig/shapefunc.png}
    \caption{形状函数$H_{\gamma}$可视化图}
    \label{fig:shapefunc}
\end{figure}

接着为了获得连续性,需要对$q_{\gamma}$施加约束,在图~\ref{fig:free_T}中,约束包括$q_b=\frac{1}{2}q_a+\frac{1}{2}q_d$, $ q_c=\frac{1}{4}q_a+\frac{3}{4}q_d$, $q_e=\frac{1}{4}q_a+\frac{1}{4}q_d+\frac{1}{2}q_i$和$q_j=\frac{1}{2}q_i+\frac{1}{2}q_k$。把这些约束代入$q_(\mathbf{x})=\sum_{\gamma} q_{\gamma}H_{\gamma}(\mathbf{x})$中,$q_b$、$q_c$、$q_e$和$q_j$就不再存在,这也意味着这些T型连接节点不参与最终的插值函数的构成。将$q_(\mathbf{x})$重新表示为入$q_(\mathbf{x})=\sum_{\eta} q_{\eta}N_{\eta}(\mathbf{x})$,这里$\eta$只包含真实节点。至此,新的形状函数$N_{\eta}(\mathbf{x})$就是最终的$C^0$连续的插值函数基,它包含了T型连接节点的权重贡献,在图~\ref{fig:free_T}中,最终的$N_{\eta}$如下:
\begin{equation}
    \begin{aligned}
        N_a &= H_a + \frac{1}{2}H_b + \frac{1}{4}H_c + \frac{1}{4}H_e,\\
        N_d &= H_d + \frac{3}{4}H_c + \frac{1}{2}H_b + \frac{1}{4}H_e,\\
        N_i &= H_i + \frac{1}{2}H_e + \frac{1}{2}H_j,\\
        N_k &= H_k + \frac{1}{2}H_j,
    \end{aligned}
\end{equation}
其他红色DOF节点的$N_{\eta}$均等于$H_{\eta}$,图~\ref{fig:finalShapefunc}给出了最终结果的可视化表示。

\begin{figure}[H]
    \centering
    \includegraphics[width=1\linewidth]{fig/final_shapefunc.png}
    \caption{解约束后形状函数$N_{\eta}$可视化图}
    \label{fig:finalShapefunc}
\end{figure}

本节介绍的带约束T型连接函数基计算方法可以从两个视角解读,既可以认为是对真实节点基函数的一种修改,引入了额外的T型连接节点影响,也可以将其视为首先计算所有节点相关的形状函数的系数,再加以约束的结果。

\subsection{多层网格插值}
在物质点法的粒子网格传输环节中,需要在粒子和网格之间传输质量、动量、速度等物理量,在粗细粒度网格交界处的物理量传输需要特别关注,本节介绍在自适应网格算法中传输物理量的方式。考虑图~\ref{fig:adaptive_transfer}中的四叉树网格结构,尽管图中左侧网格为粗粒度,为了后续的优化算法,我们假设网格依然进行了一次虚拟的划分至和右侧网格同粒度,原先的四个格点$x_{00}^{2h},x_{01}^{2h},x_{10}^{2h},x_{11}^{2h}$被拆分为$x_{00}^h,...,x_{22}^{h}$这9个格点,额外产生的虚拟的格点称为幽灵节点(Ghost Node),四叉树中本身存在的节点称为真实节点,类比上一小节中处理T型连接的算法,首先在所有细粒度网格节点(包括幽灵节点)上定义不带约束的形状函数$H_{\gamma}^h$,然后通过约束代入消除掉所有幽灵节点,可以计算到每个真实节点的带约束形状函数,比如:
\begin{equation}
    N_{11}^{2h}=H_{22}^{h}+\frac{1}{2}H_{12}^{h}+\frac{1}{2}H_{21}^{h}+\frac{1}{4}H_{11}^{h}.
\end{equation}
以质量粒子到网格传输(G2P)为例,公式为$m_i=\sum_pm_pw_{ip}$。在自适应网格算法中,对于任意粒子$p$首先找到在粒子作用域$\Omega_p$内最细粒度的网格,然后将质量传输到该粒度的网格格点上,包括一些幽灵节点。等所有粒子都传输结束后,幽灵节点会把质量传递给四叉树中的父节点,也就是更粗一级的网格,这个操作会递归进行,直到所有节点都是真实节点。

\begin{figure}[H]
    \centering
    \includegraphics[width=1\linewidth]{fig/adaptive_transfer.png}
    \caption{自适应网格上进行G2P和P2G操作}
    \label{fig:adaptive_transfer}
\end{figure}

\subsection{网格细分和粒子重采样}
本文用\(1 \leq q_g \leq Q_g\)表示网格层级,用\(1 \leq q_p \leq Q_p\)表示粒子类型,其中小写字母表示更精细的网格分辨率或更小的粒子。在预处理阶段,粒子的属性(例如位置、质量、体积和类型)由用户指定。启发式的做法是在内部使用较粗的网格分辨率(因为此处粒子分布较稀疏),而在靠近自由表面或碰撞边界处使用更精细的网格分辨率(因为此处粒子分布更密集)。静态网格自适应(例如图~\ref{fig:adaptive_cut})在预处理步骤中完成;动态网格自适应(例如图~\ref{fig:adaptive_collide_case})需要在每个时间步进行计算,网格细分和粒子重采样效果如图~\ref{fig:resample}所示。
\begin{figure}[H]
    \centering
    \includegraphics[width=1\linewidth]{fig/resample.png}
    \caption{网格细分和充采样示例图}
    \label{fig:resample}
\end{figure}

对于超弹性材料,网格和粒子的自适应完全由粒子类型决定。从最粗的层级开始,计算当前网格\(q_g\)的一个单元内最小的粒子类型\(q_p\)。如果\(q_p < q_g\),并且每个子单元的粒子数量不少于规定的每个单元的粒子数,就对该单元进行细化。对于弹塑性材料,需要进行粒子重采样。首先为每个最细粒度网格计算其到“自由表面”的曼哈顿距离\(d\)。若所有子网格到自由表面距离超过规定的距离标准时,就将它们合并为一个网格,并对粒子进行重采样。

\section{自适应DC-APIC积分器}
\subsection{自适应法线计算}
\subsection{自适应粒子网格兼容性计算}
\subsection{自适应粒子网格传输}

\section{实验结果与分析}
\subsection{实现细节}
本章节的所有实验均在Intel Core i7 processor 14700K,64G RAM,Nvidia RTX 4070Super环境进行,基于开源Taichi语言~\cite{hu2018moving}开发了自适应网格算法和DC-APIC积分器。渲染Demo使用Houdini和Arnold渲染器,首先利用Houdini内置的OpenVDB工具~\cite{museth2013vdb}中的VDB From Particles节点将粒子构建为SDF,然后中利用Convert VDB节点将SDF重建表面,然后进行PBR渲染。

\subsection{细粒度切割}
使用预先绘制的静态自适应网格可以解决传统MPM无法被小于$\Delta x$的几何体切割的问题。如图~\ref{fig:adaptive_cut}所示,一只黏稠的犰狳模型从两根相交的细金属丝间落下。图中左侧为碰撞发生前的模型;中间为根据本文方法,在金属丝附近将网格细化到基础分辨率的4倍,犰狳被正确切割;右侧为具有相近粒子数量的均匀网格,在很大程度上未能捕捉到碰撞事件。
\begin{figure}[H]
    \centering
    \includegraphics[width=1\linewidth]{fig/adaptive_cut.png}
    \caption{自适应网格上进行细线切割模拟}
    \label{fig:adaptive_cut}
\end{figure}

\subsection{弹性碰撞}
使用动态绘制的自适应网格可以捕捉到物体发生碰撞时边界的细节,而不会产生空白的缝隙,且不需要全局引入高分辨率网格。如图~\ref{fig:adaptive_collide_case}所示,在二维场景中有两个相向运动的弹性体方块,在发生接触时,接触边界的网格进行细分。图中左侧为均匀的粗网格,仿真结果中留下了很大的分离间隙。中间为一个细化了4倍的均匀网格很好地呈现出了接触情况,但代价是在内部区域进行了不必要的计算;右边为我们的自适应方案,可以兼顾计算效果和效率。
\begin{figure}[H]
    \centering
    \includegraphics[width=1\linewidth]{fig/adaptive_collide_case.png}
    \caption{自适应网格上进行物体碰撞物理模拟(二维)}
    \label{fig:adaptive_collide_case}
\end{figure}


该实验在三维场景中同样被实现(图~\ref{fig:adaptive_collide_3d}),在碰撞出自动生成了更精细的网格,可以看出本文的方法在二维和三维中都可以被正确实现且有效解决MPM固有问题。
\begin{figure}[H]
    \centering
    \includegraphics[width=1\linewidth]{fig/adaptive_coll_3d.png}
    \caption{自适应网格上进行物体碰撞物理模拟(三维)}
    \label{fig:adaptive_collide_3d}
\end{figure}

\subsection{流固耦合}
TODO

\section{本章小节}
本章聚焦于基于DC-APIC积分器的自适应广义插值物质点法,旨在攻克物质点法在物体接触、自碰撞以及边界处理等方面存在的问题。首先,系统阐述了自适应广义插值物质点法。介绍广义插值物质点法(GIMP)时,将其与传统MPM对比,明确GIMP在自适应网格方面的优势,其权重定义能满足数值稳定性需求,在均匀网格下与基于B样条的MPM等效,却更适合拓展到自适应网格。在自适应基函数部分,针对无约束T型连接负权重引发的不稳定问题,提出有约束T型连接的解决算法,从两种视角解读该算法,有效构建出连续的插值函数基。多层网格插值方面,详细说明了粗细粒度网格交界处物理量传输方式,通过定义形状函数和约束处理,完成粒子与网格间质量等物理量的准确传输。对于网格细分和粒子重采样,依据材料特性和粒子类型制定规则,实现静态和动态网格自适应。然后对自适应DC-APIC积分器展开探讨,涵盖自适应法线计算、粒子网格兼容性计算以及粒子网格传输等内容,为算法高效运行提供保障。在细粒度切割实验中,静态自适应网格成功解决传统 MPM 无法被小尺寸几何体切割的难题;弹性碰撞实验里,动态自适应网格精准捕捉物体碰撞边界细节,兼具计算效果与效率优势。综上所述,本章所提出的方法显著提升了物质点法处理复杂场景的能力,有效克服了传统方法的固有缺陷,为相关领域的数值模拟提供了更为精准、高效的解决方案,在未来有望广泛应用于更多实际场景。

% \input{chapter-ChunkCaching/S4}
%额外空白页
\clearpage
\phantom{s}
\clearpage




\addcontentsline{toc}{chapter}{参考文献}

\begin{spacing}{1.2}
\bibliographystyle{gbt7714-unsrt}
\bibliography{reference}
\end{spacing}
%\clearpage
%\phantom{s}
%\clearpage

\pagestyle{plain}\clearpage
\pagestyle{plain}
\clearpage
%\phantomsection
%\addcontentsline{toc}{chapter}{附录}
%\input {src/D1-APPENDIX.tex}
\pagestyle{plain}
\clearpage
%\phantomsection
\clearpage
\phantom{s}
\clearpage

\addcontentsline{toc}{chapter}{致谢}
{\kaishu
\chapter*{致\qquad 谢}
\begin{spacing}{1.37}
	“莫听穿林打叶声,何妨吟啸且徐行。”
\end{spacing}

\vspace{0cm} \hspace{10cm} 

\hfill  署名 \hspace{0cm}

\hfill  二零二一年十一月 \hspace{0cm}
}


%\clearpage
%\phantom{s}
%\clearpage

\pagestyle{plain}
\clearpage 
\phantomsection
\addcontentsline{toc}{chapter}{发表论文和科研情况}
\chapter*{\centering{\songti{攻读硕士学位期间发表的学术论文以及学术成果}}}
\begin{enumerate}
	\renewcommand{\labelenumi}{[\theenumi]}
	\renewcommand\baselinestretch{1}\selectfont   
	\item \textbf{Tom}, San Zhang, Si Li: The Title Of Paper[J]. The Name Of Journal, 201*, XX(X): XXXX-XXXX.
	\item XXX, \textbf{汤姆}, xxx: 论文题目[J]. 期刊名, 201*, 2*(05): xx-xx.
\end{enumerate}


\printindex
\end{document}
