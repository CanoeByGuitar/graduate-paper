\chapter{分解兼容仿射粒子网格(DC-APIC)积分器} \label{chap:dcapic}

\subsection{粒子与网格的兼容性}

本文采用了来自CPIC~\cite{hu2018moving}的兼容性概念来标记粒子与网格节点之间的关系。在CPIC中,当粒子$p$和节点$i$位于同一不被刚性碰撞物体分隔的区域内时,它们是兼容的。由于多个薄刚性物体可以将一个弹性物体分割成两个或更多区域,因此与同一网格节点兼容的两个粒子可能属于不同的区域。

在本文中,我们为每个粒子分配了一个附加属性$s$,表示从固体($s$ = 1)到流体($s$ = 0)的材料状态。我们定义粒子$p$与节点$i$兼容,当它们共享相同的固体-流体相位时,粒子$p$和节点$i$的兼容性$c_{ip}$仅当$s_p = s_i$时等于1。由于只存在固体和流体两个相位,因此与同一网格节点兼容的两个粒子必须属于相同区域。因此,不需要使用~\cite{hu2018moving}中的CDF数据结构来计算兼容性。

\begin{figure}[htbp]
    \centering
    \includegraphics[width=0.6\linewidth]{figures/phase-field-gradient.png}
    \caption{\textbf{基于相位场梯度的粒子与网格兼容性方法。} 二次B样条核范围由紫色方框标注。(a) 背景网格与B样条核的范围,其中计算的网格节点位于中心。蓝色箭头是网格节点相位值的梯度$\nabla s_i$。(b) 粒子颜色表示固体-流体相位。蓝色表示流体粒子,红色表示固体粒子。红色箭头是材料加权的粒子相位梯度值$\mathbf{Q}_p$。(c) 粒子与背景网格节点。绿色箭头表示从附近粒子计算的网格节点相位贡献估计值,通过$\nabla s_i\cdot \mathbf{Q}_p$得到。最后,通过它们所获得的总贡献的信号计算网格节点相位。(d) 计算所有网格节点的相位值。}
    \label{fig:phase-field-gradient}
\end{figure}

我们可以通过有符号距离函数(SDF)来计算网格节点的相位$s_i$,但是其时间复杂度非常高,几乎占用了每个时间步长的一半时间。因此,我们设计了一种基于相位场梯度的方法,最初用于断裂仿真~\cite{homel2017field}。首先,我们通过粒子相位状态估计网格相位梯度,计算公式为$\nabla s_i = \sum_{p} s_p \nabla w_{ip}$。其次,我们使用$\mathbf{Q}_p$表示材料加权的粒子相位梯度,$\mathbf{Q}_p = (-1)^{s_p}\sum_{i} \nabla s_i w_{ip}$。然后,我们根据粒子相位和网格节点的相位梯度的点积符号将网格节点划分为不同的相位,公式为
\begin{equation}
s_{i}=
\begin{cases} 
s_p, &  \textbf{sgn}(\nabla s_i \cdot \mathbf{Q}_p) > 0 \\
1 - s_p, & \text{otherwise}.
\end{cases}
\end{equation}
然而,两个不同相位的粒子可能会将一个网格节点划分为它们各自的相位,这意味着$\exists p, q, i,$ 使得$\textbf{sgn}(\nabla s_i \cdot \mathbf{Q}_p) > 0 \land \textbf{sgn}(\nabla s_i \cdot \mathbf{Q}_p) > 0$,其中$s_p$与$s_q$不同。这种病态情况在固体区域的锐利边界处可能频繁发生。我们采用启发式算法来解决这个问题(如图~\ref{fig:phase-field-gradient}所示)。在边界区域,粒子相位梯度在网格相位梯度方向上的投影长度$\| \nabla s_i \cdot \mathbf{Q}_p \|$越大,对网格节点相位的影响越显著。因此,我们使用梯度的点积作为确定网格节点相位的权重。\color{red}我们定义$I_p=\sum_{p}(\nabla s_i \cdot \mathbf{Q}_p)$。如果$I_p > 0$,我们将粒子$p$分类为固体粒子,并赋值$s_p=1$。反之,如果$I_p<0$,我们将粒子$p$分类为流体粒子,并赋值$s_p=0$。\color{black}

我们还实现了基于SDF的兼容性计算方法,该方法重建固体粒子等级集,并查询每个网格节点以获得其相位。这种方法比相位场梯度方法更为准确,但在构建等级集时会消耗较多时间。我们在图~\ref{fig:sdf_phase_compare}中比较了这两种方法的结果。

%-------------------------------------------------------------------------

\subsection{粒子-网格传输}
\subsubsection{DC-APIC粒子到网格传输}
\label{sec:DC-APIC_p2g}
我们使用下标$p$、$q$表示粒子量,下标$i$、$j$表示网格量。$i^{p+}$表示与粒子$p$兼容的节点,$i^{p-}$表示不兼容的节点。同样,$p^{i+}$表示与网格节点$i$兼容的粒子,$p^{i-}$表示不兼容的粒子。在大多数仿真对象中,我们仍然使用APIC传输方案。然而,在接口处,粒子仅将动量的\color{red}法向分量\color{black}传递到不兼容的网格节点,以避免在另一个相位中平滑速度,并将动量的法向分量和切向分量都传递到兼容的网格节点,以支持流固耦合过程中自动的MPM碰撞求解:
\begin{equation}
\begin{aligned}
    m_i &= \sum_{q\in\{p^{i+},p^{i-}\}}w_{iq}m_q, \\
    m_{i+} &= \sum_{q\in\{p^{i+}\}}w_{iq}m_q, \\
    (m_i\mathbf{v}_i)^{norm} &=\sum_{q\in\{p^{i+},p^{i-}\}}w_{ip}m_q(\mathbf{v}_q\mathbf{n}_q + \mathbf{B}_q\mathbf{D}_q^{-1}\Delta\mathbf{x}_{iq}), \\
    (\mathbf{m}_i\mathbf{v}_i)^{tan} &=\sum_{q\in\{p^{i+}\}}w_{iq}m_q(\mathbf{v}_q - \mathbf{v}_q\mathbf{n}_q), \\
    \mathbf{v}_i &=(m_i\mathbf{v}_i)^{norm}/m_i+(m_i\mathbf{v}_i)^{tan}/m_{i+},
\end{aligned}
\end{equation}
其中$m_i$是通过映射粒子传输到节点的总质量,$m_{i+}$仅是通过兼容粒子传输的质量,本地速度仿射项$\mathbf{B}_q\mathbf{D}_q^{-1}$有助于保持能量,否则会因如PIC等传输方案中的数值粘性而损失。我们将动量分为法向分量和切向分量,因为它们需要在速度计算步骤中除以不同的质量。具体来说,如果切向速度是$(m_i\mathbf{v}_i)^{tan}/m_i$,那么速度将被耗散。

如图~\ref{fig:p2g}所示,APIC方案在同一网格节点内分离朝相反方向运动的粒子时相对困难(在BSpline核范围内),因为粒子到网格操作会对核范围内的粒子动量进行平均,从而导致数值粘性。相反,DC-APIC将动量分解为切向分量和法向分量,仅将切向动量传递到兼容的网格节点。这避免了来自另一个相位的粒子的平均影响。此外,由于法向速度仍然按正常方式传输,并且没有直接将位置调整项添加到$\mathbf{x}_p$,它有效地解决了NFLIP~\cite{stomakhin2013material}中发现的粒子分离问题。

\begin{figure}[htbp] \centering \includegraphics[width=0.9\linewidth]{figures/p2g.png} \caption{\textbf{粒子到网格的传输对比。} (a) 传统APIC粒子到网格的传输方案。两个粒子在碰撞区域内进行平滑时,会因为核内平均而受到数值粘性影响。(b) DC-APIC粒子到网格的传输方案。法向分量传输给兼容网格节点,切向分量传输给兼容网格节点,避免了数值粘性。} \label{fig:p2g} \end{figure}


\subsubsection{DC-APIC 网格到粒子传输} \label{sec:DC-APIC_g2p}

\begin{figure*} \centering \includegraphics[width=1\linewidth]{figures/bunny3d.png} \caption{\textbf{固定的兔子与掉落的沙子。} 我们的 DC-APIC 方法能够有效解决具有自由滑移条件的流固耦合,而传统的 MPM 方法难以区分水与弹性固体兔子。} \label{fig:bunny3d} \end{figure*}

对于每个粒子,由于不连续性约束的施加,不兼容网格节点上的切向速度与粒子无关。我们采用了幽灵速度方法,在这种方法中,我们假设对于任意节点 $j\in i^{p-}$,其速度通过从粒子 $p$ 的常数外推得到,即 $v_j=v_p^n$。因此,从网格到粒子的 DC-APIC 传输方法,综合了来自兼容节点和不兼容节点的贡献,其表达式为 \begin{equation} \label{eq:DCAPIC_g2p} \begin{aligned} \mathbf{v}p &=\sum{j\in i^{p+}}w_{jp}\mathbf{v}j^* + \sum{j\in i^{p-}}w_{jp}\mathbf{v}j^*\mathbf{n}j + \sum{j\in i^{p-}}w{jp}(\mathbf{v}p-\mathbf{v}p\mathbf{n}j), \ \mathbf{B}p &= \sum{j\in i^{p+}}w{jp}\mathbf{v}j^*\mathbf{x}{jp} + \sum{j\in i^{p-}}w{jp}(\mathbf{v}_p-\mathbf{v}_p\mathbf{n}j)\Delta\mathbf{x}{ip}. \end{aligned} \end{equation} 粒子将从兼容节点获得速度,并且从不兼容节点获得法向速度分量,以实现自动 MPM 耦合而不发生自交叉。我们只在公式~\eqref{eq:DCAPIC_g2p}中使用幽灵粒子速度的切向分量,以确保稳定性。需要注意的是,这里使用的法向量必须是网格节点法向量 $\mathbf{n}_i$,以保证动量守恒。动量守恒可以在 APIC 中严格证明,其中在 G2P 和 P2G 过程前后的总粒子动量等于总网格动量。然而,DC-APIC 使用了 CPIC 技术,其中 P2G 根据流固界面的几何形状选择性地将部分动量传递到网格,并且在 G2P 过程中,通过使用 $\mathbf{v}_p^n$ 作为近似值,将失去的动量传回粒子。从另一个角度来看,我们的传输方案可以视为两种操作的组合:正交分解和 CPIC。显然,正交分解保持了总动量。在 CPIC 的 P2G 过程中,速度被传递到兼容节点。然后,在 G2P 过程中,不兼容节点将幽灵速度传回粒子。这些幽灵速度与前一时刻的粒子速度相同。因此,整个 P2G-G2P 流程看起来像是在 APIC 滤波器与真实数据之间进行插值,其中真实数据指的是速度与前一时刻保持一致,从而确保动量守恒。关于能量守恒的影响的更详细分析将在节~\ref{sec:results}中给出。

\subsection{带有 DC-APIC 方案的 MPM 工作流程}
\begin{figure*}[htbp]
\centering
\includegraphics[width=0.9\linewidth]{figures/pipeline.png}
\caption{\textbf{我们算法的数据流}。图示了在一个时间步内粒子与网格之间的数据流。红色实线表示 DC-APIC 方案的额外步骤。 \textbf{界面信息计算(IIC)}步骤在每个时间步的开始计算粒子法向量 $n_p$、网格节点法向量 $n_i$ 和网格节点的固流相场状态 $s_i$,这些信息将用于 DC-APIC 粒子到网格(P2G)和网格到粒子(G2P)步骤。} 
\label{fig:pipeline}
\end{figure*}

与传统的 MPM 工作流程相比,我们增加了一个额外的步骤来获取粒子与网格的兼容性信息,以调节固流耦合中网格到粒子和粒子到网格的过程,处理材料间的不连续性。图~\ref{fig:pipeline} 展示了我们数值求解器的主要数据流,主要工作流程如下所列:

\begin{enumerate}
\item[(1)] \textbf{界面信息计算。} 计算粒子法向量 $n_p$ 和网格节点法向量 $n_i$ 以及网格上的固流相场状态 $s_i$,这些信息用于控制 DC-APIC 的粒子到网格和网格到粒子步骤。
\item[(2)] \textbf{粒子到网格。} 根据 DC-APIC 方案将质量和动量从粒子传递到网格,具体细节见第~\ref{sec:DC-APIC_p2g} 节。
\item[(3)] \textbf{网格速度更新。}
根据邻近粒子的变形梯度 $\mathbf{F}_p^n$ 计算施加于节点 $i$ 上的力 $\mathbf{f}_i^n=-\sum_p V_p^0 \frac{\partial \Psi}{\partial F} {\mathbf{F}_p^n}^T \nabla w_{ip}^n$,然后以辛欧拉方式更新速度:$\mathbf{v}_i^{n+1}=\mathbf{v}_i^n + \Delta t \mathbf{f}_i^n / m_i$。
\item[(4)] \textbf{网格到粒子。}
根据第~\ref{sec:DC-APIC_g2p} 节的 APIC 方法更新速度 $\mathbf{v_p^{n+1}}$、粒子的空间速度梯度 $(\nabla \mathbf{v})_p^{n+1}$ 和仿射状态 $\mathbf{B}_p^{n+1}$,然后进行粒子推进与碰撞处理。
\item[(5)] \textbf{应变更新。}
更新粒子的试验变形梯度:$\mathbf{F}_p^{n+1} = (I + \Delta t \nabla \mathbf{v}_p^{n+1}) \mathbf{F}_p^n$。

\end{enumerate}

\begin{figure}[htbp]
    \centering
    \includegraphics[width=0.7\linewidth]{figures/sdf_phase_compare.png}
    \caption{\textcolor{red}{\textbf{水滴在弹性球上}。 (顶部) 我们基于相场梯度兼容性计算的 DC-APIC MPM。水容易从自由滑移的弹性球上分离。 (中部) 传统 MPM 模拟的耦合存在不真实的粘度,水很难离开弹性体。 (底部) 基于 SDF 的 DC-APIC 重建了水平集以分类网格节点的相态,其结果显然与相场梯度相同,但计算时长较长。}}
    \label{fig:sdf_phase_compare}
\end{figure}

\subsubsection{界面信息计算}
\label{interface_information_calculation}
在 \textbf{IIC} 步骤的开始,我们使用额外的 P2G 步骤仅传递相场。接收到流体和固体相的网格节点被标记为边界节点,映射到这些节点的粒子被标记为边界粒子。

\subsubsection{法向量估计}
传统的有符号距离函数(SDF)方便执行内外查询和法向量估计。然而,每次时间步重建粒子水平集是非常耗时的。我们采用与~\cite{fang2020iq} 相同的方法,通过选择固体粒子的负质量梯度场来确定固体的法向量。
\begin{equation}
\begin{aligned}
m_i^s &= \sum_{p^s} m_p w_{ip},\\
\mathbf{n}_p^s &= -\sum_i m_i^s \nabla w_{ip} / \| \sum_i m_i^s \nabla w_{ip} \| ,\\
\end{aligned}
\end{equation}
其中 $m_i^s$ 是映射到一个网格节点的固体质量,$p^s$ 表示固体粒子,$\mathbf{n}_p^s$ 表示固体粒子的法向量。对于网格法向量和流体粒子的法向量,我们设计了一种不同的策略,如下所示:
\begin{equation}
\label{eq:ni_npf}
\begin{aligned}
\mathbf{n}_i &= \sum_{p^s} \mathbf{n}_p^s w_{ip} / \| \sum_{p^s} \mathbf{n}_p^s w_{ip} \| ,\\
\mathbf{n}_p^f &= -\sum_i \mathbf{n}_i w_{ip} / \| \sum_i \mathbf{n}_i w_{ip} \|,\\
\end{aligned}
\end{equation}
其中 $\mathbf{n}_i$ 是网格节点的法向量,可以覆盖所有界面网格节点,$\mathbf{n}_p^f$ 表示流体粒子的法向量。从 Eq.~\eqref{eq:ni_npf} 可以看出,我们首先使用 B 样条平滑核对固体粒子法向量 $\mathbf{n}_p^s$ 进行平滑,以得到网格节点的法向量。平滑核可以使 DC-APIC 的网格到粒子(第~\ref{sec:DC-APIC_p2g} 节)和粒子到网格(第~\ref{sec:DC-APIC_g2p} 节)更加稳定。然后,我们利用网格节点法向量来计算流体粒子的法向量。该策略既稳定又高效,避免了在计算流体法向量时需要通过空间数据结构查找最近的固体粒子。

\begin{figure}[htbp]
    \centering
    \includegraphics[width=0.8\linewidth]{figures/duck.png}
    \caption{\textbf{不同密度的橡胶鸭子玩具掉入水中。} 从左到右,鸭子与水的密度比为 0.5,1.0 和 2.5。}
    \label{fig:duck}
\end{figure}