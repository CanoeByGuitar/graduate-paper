\chapter{分解兼容仿射粒子网格(DC-APIC)积分器} \label{chap:dcapic}
针对物质点法在进行物理仿真时存在的难以约束自由滑动边界条件的问题,本章节提出了分解兼容仿射粒子网格(DC-APIC)积分器算法,有效地解决传统MPM算法中数值粘性问题。本章首先提出粒子与网格兼容性的定义,并给出两种计算方式:基于符号距离函数场(SDF)的方法和基于相场梯度的方法;接着分辨阐述DC-APIC粒子到网格传输算法和网格到粒子传输算法,详细给出了质量和动量的传输公式,并论述了该方法维持动量守恒的特性;然后给出结合DC-APIC的物质点法的算法全流程,着重描述了在处理交接面几何边界时采用的不同法线计算方式;最后在多个二维和三维的实验中对比了本文同其他传统传输方式的优点,并罗列了时间开销。

\section{粒子与网格的兼容性}
本文采用了来自兼容性粒子网格方法(Compatible Particle in Cell, CPIC)~\cite{hu2018moving}中的兼容性概念来标记粒子与网格节点之间的关系。在CPIC中,当粒子$p$和节点$i$位于同一不被刚性碰撞物体分隔的区域内时,它们是兼容的;反之则为不兼容的。由于多个薄刚性物体可以将一个弹性物体分割成两个或更多区域,因此与同一网格节点兼容的两个粒子可能属于不同的区域。

在本文中,我们为每个粒子分配了一个附加属性$s$,表示从固体($s$ = 1)到流体($s$ = 0)的材料状态。我们定义粒子$p$与节点$i$为兼容状态当且仅当它们相位相同,即粒子$p$和节点$i$间的兼容性$c_{ip}$当前仅当$s_p = s_i$时为1。由于只存在固体和流体两个相位,与同一网格节点兼容的两个粒子必须属于相同区域。因此,不需要使用~\cite{hu2018moving}中的颜色标记距离向量场(Colored Distance Field, CDF)数据结构来计算兼容性。

\begin{figure}[htbp]
    \centering
    \includegraphics[width=0.6\linewidth]{figures/phase-field-gradient.png}
    \caption{基于相场梯度的粒子与网格兼容性计算方法流程}
    \label{fig:phase-field-gradient}
\end{figure}

根据粒子位置,使用各向异性核~\cite{yu2013reconstructing}可以平滑地重建出符号距离函数场(Signed Distance Field, SDF),进而比较网格格点的符号距离函数值来计算网格节点的相位$s_i$,但是其时间复杂度非常高,在场景测试中几乎占用了每个时间步长的一半时间。因此,我们设计了一种基于相场梯度的方法,该方法最早用于断裂仿真~\cite{homel2017field}。首先,通过粒子相位状态估计网格相位梯度,计算公式为$\nabla s_i = \sum_{p} s_p \nabla w_{ip}$。其次,使用$\mathbf{Q}_p$表示材料加权的粒子相位梯度,$\mathbf{Q}_p = (-1)^{s_p}\sum_{i} \nabla s_i w_{ip}$。然后,根据粒子相位和网格节点的相位梯度的点积符号将网格节点划分为不同的相位,公式为
\begin{equation}
s_{i}=
\begin{cases} 
s_p, &  \textbf{sgn}(\nabla s_i \cdot \mathbf{Q}_p) > 0 \\
1 - s_p, & \text{otherwise}.
\end{cases}
\end{equation}
然而,两个不同相位的粒子可能会将一个网格节点划分为它们各自的相位,这意味着$\exists p, q, i,$ 使得$\textbf{sgn}(\nabla s_i \cdot \mathbf{Q}_p) > 0 \land \textbf{sgn}(\nabla s_i \cdot \mathbf{Q}_p) > 0$,其中$s_p$与$s_q$不同。这种病态情况在固体区域的锐利边界处可能频繁发生。我们采用启发式算法来解决这个问题(如图~\ref{fig:phase-field-gradient}所示)。在边界区域,粒子相位梯度在网格相位梯度方向上的投影长度$\| \nabla s_i \cdot \mathbf{Q}_p \|$越大,对网格节点相位的影响越显著。因此,我们使用梯度的点积作为确定网格节点相位的权重。我们定义$I_p=\sum_{p}(\nabla s_i \cdot \mathbf{Q}_p)$。如果$I_p > 0$,我们将粒子$p$分类为固体粒子,并赋值$s_p=1$。反之,如果$I_p<0$,我们将粒子$p$分类为流体粒子,并赋值$s_p=0$。具体流程如图~\ref{fig:phase-field-gradient}所示,二次B样条核范围由紫色方框标注。图中(a)展示了背景网格与B样条核的范围,其中计算的网格节点位于中心。蓝色箭头是网格节点相位值的梯度$\nabla s_i$;(b) 中粒子颜色表示固体-流体相位。蓝色表示流体粒子,红色表示固体粒子。红色箭头是材料加权的粒子相位梯度值$\mathbf{Q}_p$。(c)中标明了粒子与背景网格节点。绿色箭头表示从附近粒子计算的网格节点相位贡献估计值,通过$\nabla s_i\cdot \mathbf{Q}_p$得到,通过它们所获得的总贡献的信号计算网格节点相位。(d)为最后一步,计算所有网格节点的相位值。



%-------------------------------------------------------------------------

\section{粒子-网格传输算法}
\subsection{DC-APIC粒子到网格传输算法}
\label{sec:DC-APIC_p2g}
本章使用下标$p$、$q$表示粒子,下标$i$、$j$表示网格。$i^{p+}$表示与粒子$p$兼容的节点,$i^{p-}$表示不兼容的节点。同样,$p^{i+}$表示与网格节点$i$兼容的粒子,$p^{i-}$表示不兼容的粒子。本文方法中,在大多数仿真区域内部仍然使用传统APIC传输方案,而在接口处,粒子仅将动量的法向分量传递到不兼容的网格节点,以避免在另一个相位中平滑速度,同时粒子将动量的法向分量和切向分量都传递到兼容的网格节点,以支持流固耦合过程中自动的MPM碰撞求解,公式如下:
\begin{equation}
\begin{aligned}
    m_i &= \sum_{q\in\{p^{i+},p^{i-}\}}w_{iq}m_q, \\
    m_{i+} &= \sum_{q\in\{p^{i+}\}}w_{iq}m_q, \\
    (m_i\mathbf{v}_i)^{norm} &=\sum_{q\in\{p^{i+},p^{i-}\}}w_{ip}m_q(\mathbf{v}_q\mathbf{n}_q + \mathbf{B}_q\mathbf{D}_q^{-1}\Delta\mathbf{x}_{iq}), \\
    (\mathbf{m}_i\mathbf{v}_i)^{tan} &=\sum_{q\in\{p^{i+}\}}w_{iq}m_q(\mathbf{v}_q - \mathbf{v}_q\mathbf{n}_q), \\
    \mathbf{v}_i &=(m_i\mathbf{v}_i)^{norm}/m_i+(m_i\mathbf{v}_i)^{tan}/m_{i+},
\end{aligned}
\end{equation}
其中$m_i$是通过映射粒子传输到节点的总质量,$m_{i+}$仅是通过兼容粒子传输的质量,本地速度仿射项$\mathbf{B}_q\mathbf{D}_q^{-1}$有助于保持能量,否则会因如PIC等传输方案中的数值粘性而损失。我们将动量分为法向分量和切向分量,因为它们需要在速度计算步骤中除以不同的质量。具体来说,如果切向速度是$(m_i\mathbf{v}_i)^{tan}/m_i$,那么速度将被耗散。

如图~\ref{fig:p2g}所示,APIC方案在同一网格节点内分离朝相反方向运动的粒子时相对困难(在BSpline核范围内),因为粒子到网格操作会对核范围内的粒子动量进行平均,从而导致数值粘性。相反,DC-APIC将动量分解为切向分量和法向分量,仅将切向动量传递到兼容的网格节点。这避免了来自另一个相位的粒子的平均影响。此外,由于法向速度仍然按正常方式传输,并且没有直接将位置调整项添加到$\mathbf{x}_p$,它有效地解决了NFLIP~\cite{stomakhin2013material}中发现的粒子穿透问题。

\subsection{DC-APIC网格到粒子传输算法} \label{sec:DC-APIC_g2p}
对于每个粒子,由于不连续性约束的施加,不兼容网格节点上的切向速度与粒子无关。我们采用了幽灵粒子速度方法,在这种方法中,我们假设对于任意节点 $j\in i^{p-}$,其速度通过从粒子$p$的上一时间步速度常数外推得到,即 $v_j=v_p^n$。因此,从网格到粒子的 DC-APIC传输方法,综合了来自兼容节点和不兼容节点的贡献,其表达式为:
\begin{equation}
    \label{eq:DCAPIC_g2p}
    \begin{aligned}
        \mathbf{v}_p &=\sum_{j\in i^{p+}}w_{jp}\mathbf{v}_j^* + \sum_{j\in i^{p-}}w_{jp}\mathbf{v}_j^*\mathbf{n}_j + \sum_{j\in i^{p-}}w_{jp}(\mathbf{v}_p-\mathbf{v}_p\mathbf{n}_j), \\
        \mathbf{B}_p &= \sum_{j\in i^{p+}}w_{jp}\mathbf{v}_j^*\mathbf{x}_{jp} + \sum_{j\in i^{p-}}w_{jp}(\mathbf{v}_p-\mathbf{v}_p\mathbf{n}_j)\Delta\mathbf{x}_{ip}.
    \end{aligned}
\end{equation}

粒子将从兼容节点获得速度,并且从不兼容节点获得法向速度分量,以实现自动MPM耦合而不发生自穿插。我们只在公式~\eqref{eq:DCAPIC_g2p}中使用幽灵粒子速度的切向分量,以确保稳定性。需要注意的是,这里使用的法向量必须是网格节点法向量 $\mathbf{n}_i$,以保证动量守恒。动量守恒可以在APIC积分器中严格证明,其中在G2P和P2G过程前后的总粒子动量等于总网格动量。然而,DC-APIC算法使用了CPIC技术,其中P2G根据流固界面的几何形状选择性地将部分动量传递到网格,并且在G2P过程中,通过使用 $\mathbf{v}_p^n$ 作为近似值,将失去的动量传回粒子。从另一个角度来看,本文的传输方案可以视为两种操作的组合:正交分解和 CPIC。显然,正交分解保持了总动量;在CPIC的P2G过程中,速度被传递到兼容节点。然后,在G2P过程中,不兼容节点将幽灵速度传回粒子。这些幽灵速度与前一时刻的粒子速度相同。因此,整个P2G-G2P流程看起来像是在APIC积分器与真实数据之间进行插值,其中真实数据指的是速度与前一时刻保持一致,从而确保动量守恒。

\section{基于DC-APIC传输算法的MPM工作流程}
\begin{figure*}[htbp]
\centering
\includegraphics[width=0.9\linewidth]{figures/pipeline.png}
\caption{基于DC-APIC积分器的MPM算法流程} 
\label{fig:pipeline}
\end{figure*}

与传统的MPM工作流程相比,本文增加了一个额外的步骤来获取粒子与网格的兼容性信息,以调节固流耦合中网格到粒子和粒子到网格的传输过程,处理材料间的不连续性。图~\ref{fig:pipeline} 展示了我们数值求解器的主要数据流,其主要工作流程如下所列:

\begin{enumerate}
\item[(1)] \textbf{界面信息计算。} 计算粒子法向量 $n_p$ 和网格节点法向量 $n_i$ 以及网格上的流固相场状态 $s_i$,这些信息用于控制 DC-APIC 的粒子到网格和网格到粒子步骤。
\item[(2)] \textbf{粒子到网格传输。} 根据 DC-APIC方案将质量和动量从粒子传递到网格,具体细节参见第~\ref{sec:DC-APIC_p2g}节。
\item[(3)] \textbf{网格速度更新。}
根据邻近粒子的变形梯度 $\mathbf{F}_p^n$ 计算施加于节点 $i$ 上的力 $\mathbf{f}_i^n=-\sum_p V_p^0 \frac{\partial \Psi}{\partial F} {\mathbf{F}_p^n}^T \nabla w_{ip}^n$,然后以显式欧拉方式更新速度:$\mathbf{v}_i^{n+1}=\mathbf{v}_i^n + \Delta t \mathbf{f}_i^n / m_i$。
\item[(4)] \textbf{网格到粒子传输。}
根据第~\ref{sec:DC-APIC_g2p}节的 APIC 方法更新速度 $\mathbf{v_p^{n+1}}$、粒子的空间速度梯度 $(\nabla \mathbf{v})_p^{n+1}$ 和仿射状态 $\mathbf{B}_p^{n+1}$,然后进行粒子推进与碰撞处理。
\item[(5)] \textbf{应变更新。}
更新粒子的试验形变梯度:$\mathbf{F}_p^{n+1} = (I + \Delta t \nabla \mathbf{v}_p^{n+1}) \mathbf{F}_p^n$。

\end{enumerate}



\subsection{界面信息计算}
\label{interface_information_calculation}
在 \textbf{IIC} 步骤的开始,我们使用额外的 P2G 步骤仅传递相场。接收到流体和固体相的网格节点被标记为边界节点,映射到这些节点的粒子被标记为边界粒子。

\subsection{法向量估计}
传统的有符号距离函数(SDF)方便执行内外查询和法向量估计。然而,每次时间步重建粒子水平集是非常耗时的。我们采用与~\cite{fang2020iq} 相同的方法,通过选择固体粒子的负质量梯度场来确定固体的法向量。
\begin{equation}
\begin{aligned}
m_i^s &= \sum_{p^s} m_p w_{ip},\\
\mathbf{n}_p^s &= -\sum_i m_i^s \nabla w_{ip} / \| \sum_i m_i^s \nabla w_{ip} \| ,\\
\end{aligned}
\end{equation}
其中 $m_i^s$ 是映射到一个网格节点的固体质量,$p^s$ 表示固体粒子,$\mathbf{n}_p^s$ 表示固体粒子的法向量。对于网格法向量和流体粒子的法向量,我们设计了一种不同的策略,如下所示:
\begin{equation}
\label{eq:ni_npf}
\begin{aligned}
\mathbf{n}_i &= \sum_{p^s} \mathbf{n}_p^s w_{ip} / \| \sum_{p^s} \mathbf{n}_p^s w_{ip} \| ,\\
\mathbf{n}_p^f &= -\sum_i \mathbf{n}_i w_{ip} / \| \sum_i \mathbf{n}_i w_{ip} \|,\\
\end{aligned}
\end{equation}
其中 $\mathbf{n}_i$ 是网格节点的法向量,可以覆盖所有界面网格节点,$\mathbf{n}_p^f$ 表示流体粒子的法向量。从 Eq.~\eqref{eq:ni_npf} 可以看出,我们首先使用 B 样条平滑核对固体粒子法向量 $\mathbf{n}_p^s$ 进行平滑,以得到网格节点的法向量。平滑核可以使 DC-APIC 的网格到粒子(第~\ref{sec:DC-APIC_p2g} 节)和粒子到网格(第~\ref{sec:DC-APIC_g2p} 节)更加稳定。然后,我们利用网格节点法向量来计算流体粒子的法向量。该策略既稳定又高效,避免了在计算流体法向量时需要通过空间数据结构查找最近的固体粒子。


\section{实验结果与分析}
本章节的所有实验均在Intel Core i7 processor 14700K,64G RAM环境进行,基于开源框架Ziran~\cite{fang2020iq}开发了DC-APIC积分器。渲染Demo使用Houdini和Arnold渲染器,首先利用Houdini内置的OpenVDB工具~\cite{museth2013vdb}中的VDB From Particles节点将粒子构建为SDF,然后中利用Convert VDB节点将SDF重建表面,然后进行PBR渲染。

通过将本章节提出的 DC-APIC积分器结合MPM算法流程,实现CPU端仿真程序,并进行对比实验来验证该算法的有效性。具体的实验场景参数设置见表\ref{tab:demo_parameter}。
\begin{table*}[h]
    \centering
    \caption{\textbf{Parameters and performance.}}
    \resizebox{\textwidth}{!}{%
    \begin{tabular}{l@{\hspace{1cm}}|c|c|c|c|c|c|c}
    \hline
    Example & 
    sec/frame & 
    $\Delta x$ &
    $\Delta t$ &
    \#Particles &
    Young's Modulus &
    Poisson's Ratio &
    density ratio
    \\
    \hline
    \makebox[7em][l]{(Fig.~\ref{fig:bunny3d} left)} Pinned bunny & 6.33  & 0.1 & 0.001 & 98K & 8000 & 0.4  & 1.0\\
    \makebox[7em][l]{(Fig.~\ref{fig:bunny3d} right)} Pinned bunny & 3.60  & 0.1 & 0.001 & 98K & 8000 & 0.4  & 1.0\\
    \makebox[7em][l]{(Fig.~\ref{fig:bunny_bath})} Bear bath & 24.17 & 0.1 & $5 \times 10^{-4}$ & 400K & 6000 & 0.4 & 1.2 \\
    \makebox[7em][l]{(Fig.~\ref{fig:duck})} Duck & 57.63 & 0.1 & 0.001 & 640K & 6000 & 0.4 & 0.5/1.0/2.5 \\
    
    \makebox[7em][l]{(Fig.~\ref{fig:jelly} left)} Jelly Sand & 14.72 & 0.05 & $5 \times 10^{-4}$ & 1.2M & 1500 & 0.2 & 1.0 \\
    \makebox[7em][l]{(Fig.~\ref{fig:jelly} middle)} Jelly Sand & 17.16 & 0.05 & $5 \times 10^{-4}$ & 1.2M & 4000 & 0.3 & 1.0 \\
    \makebox[7em][l]{(Fig.~\ref{fig:jelly} right)} Jelly Sand & 15.91 & 0.05 & $5 \times 10^{-4}$ & 1.2M & 6000 & 0.4 & 1.0 \\
    \makebox[7em][l]{(Fig.~\ref{fig:sdf_phase_compare} top)} 
     Balls(2D) & 0.78 & 0.05 & 0.001 & 8K & 4000 & 0.4 & 1.0 \\
    \makebox[7em][l]{(Fig.~\ref{fig:sdf_phase_compare} middle)} 
     Balls(2D) & 0.42 & 0.05 & 0.001 & 8K & 4000 & 0.4 & 1.0 \\
    \makebox[7em][l]{(Fig.~\ref{fig:sdf_phase_compare} bottom)} 
     Balls(2D) & 1.14 & 0.05 & 0.001 & 8K & 4000 & 0.4 & 1.0 \\
    \hline
    \end{tabular}%
    }
    
    \label{tab:demo_parameter}
\end{table*}

\begin{figure}[htbp]
    \centering
    \includegraphics[width=0.7\linewidth]{figures/sdf_phase_compare.png}
    \caption{二维水球坠落实验:两种计算兼容性方法的DC-APIC算法和APIC算法的对比}
    \label{fig:sdf_phase_compare}
\end{figure}

图~\ref{fig:sdf_phase_compare}中通过二维水块落在弹性软体球上的实验,比较了基于粒子重建SDF和基于相场梯度这两种方法计算粒子网格兼容性的效果对比。使用本文提出的基于相场梯度计算兼容性的方法可以取得SDF方法类似的效果,并且开销更小。图中还额外对比了APIC积分器的效果,其模拟的耦合存在不真实的粘度,水很难离开弹性体;而另外两种使用DC-APIC积分器的方法都使得水能够自由滑落球面。


\begin{figure}[H]
    \centering 
    \includegraphics[width=1\linewidth]{figures/bunny3d.png} 
    \caption{三维沙堆砸落碰撞弹性兔子并分离实验} 
    \label{fig:bunny3d} 
\end{figure}

图~\ref{fig:bunny3d}展示了三维场景下,本文的DC-APIC方法能够有效解决具有自由滑移条件的流固耦合边界条件,而传统的基于DC-APIC积分器的MPM方法难以分离沙粒流与弹性固体兔子。

\begin{figure}[H]
    \centering
    \includegraphics[width=0.8\linewidth]{figures/duck.png}
    \caption{流固耦合固体密度测试}
    \label{fig:duck}
\end{figure}
图~\ref{fig:duck}通过不同固体与流体密度比,测试了本文DC-APIC方法能够正确处理浮力,实现物理正确的固体落水的漂浮、悬浮或沉底效果。

\begin{figure}[H]
    \centering
    \includegraphics[width=0.7\linewidth]{figures/jelly_6.png}
    \caption{三维不同材料果冻同颗粒流交互实验}
    \label{fig:jelly}
\end{figure}
图~\ref{fig:jelly}展示了果冻材料的固体同无粘性颗粒流的交互动画,不同颜色的果冻有不同的物理参数,表现出不一样的最大形变量和弹性系数,颗粒流在最后均可以顺利同果冻分离而不发生粘滞现象。

\begin{figure}[H]
    \centering
    \includegraphics[width=0.7\linewidth]{figures/bunny_bath.png}
    \caption{溃坝冲击三维实验}
    \label{fig:bunny_bath}
\end{figure}
在图~\ref{fig:bunny_bath}中,水体溃坝并击倒一只由超弹性材料构成的小熊玩具并将其冲走,模拟实验中水粒子正确地绕过弹性表面流动,没有产生非物理的粘性。

\begin{figure}[H]
    \centering
    \includegraphics[width=1\linewidth]{figures/echarts.png}
    \caption{DC-APIC算法、APIC算法和IQ-MPM算法的时间复杂度对比}
    \label{fig:echarts}
\end{figure}
在上文的众多实验中,流体与固体相互作用发生前,DC-APIC算法、APIC算法和IQ-MPM算法的平均计算效率非常相似。然而,在相互作用发生后,IQ-MPM需要大量时间计算流体-固体边界上的压力,这导致其处理时间明显长于传统MPM。相比之下,DC-APIC使用统一的背景网格,不需要额外的求解器,因此其计算速度与传统MPM相当。三种传输方案的平均时间消耗(秒/帧)对比如图~\ref{fig:echarts}所示。

\begin{figure}[H]
    \centering
    \includegraphics[width=0.8\linewidth]{figures/energy.png}
    \caption{粒子动能变化曲线实验}
    \label{fig:energy}
\end{figure}
图~\ref{fig:energy}中展示了模拟过程中,粒子的总动能变化曲线,在三维鸭子落水的实验中,流体撞击固体后,APIC方法中的动能比DC-APIC方法中的动能更快地耗散。


\section{本章小节}
本章介绍了本文的核心创新点:分解兼容仿射粒子网格(DC-APIC)积分器。该积分器通过在兼容条件和非兼容条件下传递不同的动量,既能实现流固边界的顺利分离,又保留了传统MPM方法可以自动处理碰撞、不发生物体穿透现象的优点。本章节提供的结合DC-APIC的MPM方法,能够正确处理流固界面演变、固体受浮力影响动画,在二维和三维的多个实验中展现出鲁棒性和优秀的性能表现,结合后文的自适应网格算法,本文将进一步增加流固耦合交界面的细节真实性并优化性能表现。