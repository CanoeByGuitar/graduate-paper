\chapter{绪论} \label{chap:introduction}
图形学物理模拟是计算机图形学中的一个重要研究方向,其核心目标是通过计算机模拟物理现象,生成真实感强的动态图像和动画效果。随着计算机硬件的不断进步和图形处理能力的提升,图形学物理模拟在多个领域的应用逐渐拓展,尤其在游戏开发、电影制作、虚拟现实(VR)和增强现实(AR)等行业中发挥着关键作用。在游戏开发中,物理模拟涉及碰撞检测、实时流体解算、实时布料解算、实时刚体解算、角色布娃娃系统等,极大提升了游戏的沉浸感和互动性。在电影制作中,物理模拟帮助特效团队生成复杂的自然现象,如爆炸、火焰、烟雾和水流等,增强了视觉效果的真实性。此外,物理模拟还在工程仿真、医学模拟、教育培训等领域得到了广泛应用,为各行各业提供了更为精准的分析和预测工具。现代应用程序中的许多技术依赖于固体与流体之间的动态交互,例如充气橡胶轮胎、液压系统、飞行器、漂浮的船只、风车等。随着虚拟手术、数字化制造和软体机器人等新兴应用的出现,流固耦合的快速计算方法的需求也日益增加。这些方法不仅能够提升虚拟环境中的真实感,还能在许多实际应用中提供高效的解决方案,尤其是在需要精确模拟和实时交互的场合。随着相关技术的不断发展,如何实现高效、精准且具有高逼真度的流固耦合仿真,成为了当前物理模拟领域中的一项重要挑战。本章节将从研究背景,研究现状,研究目标这三个方面对物理仿真中的流固耦合和物质点法进行全面的介绍,之后总结了本文的研究目标与贡献,最后给出了全文的组织结构。

\section{研究背景与意义}\label{sect:intro:bg}
在物理模拟中,拉格朗日视角(Lagrangian perspective)和欧拉视角(Eulerian perspective)是两种常见的描述物理现象的方法,它们在处理流体动力学和固体力学等问题时具有各自的优势和应用场景。拉格朗日视角以物体的运动为核心,物体被离散为很多个物质点,每个物质点代表了物体在某一小空间范围内的物质属性,在算法流程中关注物质点随时间的轨迹和变形。通过跟踪每个物质点的运动状态,拉格朗日视角能够直接描述物体内部的变形、应力和速度等物理量。欧拉视角则固定一个观察区域,并关注物体在该区域内的性质变化。在欧拉方法中,计算网格是固定的,物质随着时间流动经过网格,模拟的焦点是物质通过固定空间的运动和变形。拉格朗日视角可以轻松追踪每个物质点的运动轨迹,因此在模拟大尺度的软体形变和流体拓扑变化方面有较大优势,但在计算物理量梯度时需要搜集邻居物质点的信息,计算开销比较昂贵;欧拉视角由于计算网格固定,通过有限差分法可以方便计算物理量梯度,但在描述物体拓扑变化方面比较困难。因此通过网格法做出的流体模拟效果往往在视觉表现上更像比较黏稠的果冻。混合拉格朗日/欧拉方法可以综合两者的优点,在网格上计算物理量梯度,把计算结果插值回粒子后,继续追踪粒子的位置变化。

\begin{figure}[htbp]
    \centering
    \includegraphics[width=1\linewidth]{fig/mpm_cases.png}
    \caption{物质点法的模拟场景}
    \label{fig:mpm-cases}
\end{figure}

在实践中,往往根据物体材料的特性选用不用的模拟视角。例如,对于烟雾模拟常常采用欧拉视角进行模拟,对空间区域进行网格划分,追踪每个网格烟雾密度的变化;对于拓扑变化频繁的液体,在仿真过程中,其形态可能在液块、液面、液滴之间切换,所以常常用拉格朗日视角方法,比如:光滑粒子流体动力学方法(SPH)或者混合拉格朗日/欧拉方法,比如:粒子网格法(PIC)、流体隐式粒子法(FLIP);对于可变形固体,往往采用欧拉视角的可变形网格方法,比如有限元方法(FEM),或者混合拉格朗日/欧拉方法的物质点法(MPM),它们都基于本构模型对材料物理方程的弱形式进行求解。

近年来,物质点法(MPM)得到了大量研究,在模拟真实世界中取得了前所未有的效果。物质点法是一种混合拉格朗日/欧拉方法,可以作为流体、气体、弹性固体、刚体等材料的统一模拟框架,同时也支持和其他欧拉或拉格朗日方法进行耦合。物质点法具备以下优点:其一,物质点法可以从动量守恒的弱形式推导出来,从而实现物理定律的精确离散化;其二,由于网格和粒子间的相互插值,边界条件、材料碰撞等模拟中需要特别关注的部分可以不用额外代价、自然地被解决;其三,通过赋予粒子不同的材料属性或本构模型,可以轻松实现自动的多材料和多相耦合。如今,物质点法已经可以应用在绝大多数场景,模拟大量不同的材料,包括水、蜂蜜、牛奶、非牛顿流体、沙子、雪、布料、果冻、橡胶、汽车、肉类、木头、海绵等等。图~\ref{fig:mpm-cases}展示了物质点法模拟各种真实场景的强大能力。

\begin{figure}[htbp]
    \centering
    \includegraphics[width=1\linewidth]{fig/houdini.png}
    \caption{节点物理仿真软件Houdini的操作界面}
    \label{fig:houdini}
\end{figure}

然而,现有的物质点法在算法层面和工程角度仍有优化空间。一方面,物质点法算法本身在模拟物体碰撞、流固耦合时会产生数值耗散,造成物理非真实感的缺陷。此外,因为碰撞的判定是基于网格的,受网格分辨率影像,在物体碰撞边界处会出现空白缝隙这一非真实效果。另一方面,在国内游戏影视和工业生产领域,缺少类似于国外开发的Houdini这种基于节点系统的物理仿真软件。节点系统是国外顶级影视特效团队多年工作中打磨出来的现代化特效生产工作流,可以视作一种可视化编程方法,节点连线便于特效师、动画师等非程序开发者进行艺术效果打磨;节点本身可以在程序测由专业的图形程序来实现。图~\ref{fig:houdini}展示了Houdini的节点系统示意图和效果。

因此,本文将基于物质点法求解框架,继续研究算法优化细节和配套工程。随着国内游戏和影视领域越来越追求高品质动画、高真实感沉浸式体验,如何利用物质点法打磨更精细的物理动画效果以及加速特效领域工业流水线的建设成为了难题,面临的主要挑战有:

\begin{enumerate}
    \item[1.] 在物质点法算法流程中涉及两次粒子网格的信息交换,包括粒子到网格(P2G)和网格到粒子(G2P),由于粒子数量一般多于网格点数量,因此导致了自由度的丢失,在模拟液体时会出现数值粘性问题,这个问题在模拟流固耦合中尤其严重,常常出现非粘性的流体经过软体表面时被固体粒子捕获住难以顺畅流过和分离的现象。
    \item[2.] 在物质点法中,两个粒子如果映射到同一个网格点上,就会判定为接触,进而计算相互的物理影响,因此在解决自碰撞问题时,天然存在一个网格边长的分离距离,导致在网格分辨率不高时,相互碰撞的物体可以肉眼观察到一条空白的缝隙。
    \item[3.] 影视特效行业有迫切使用物质点法的需求来制作各种材料物体的交互,但国产市场缺少现代化的基于节点的物理仿真软件。节点化系统通过将算法流程的特定步骤抽象为节点,可以同其他节点进行上下游的通信,例如将物质点法中的粒子到网格传输(P2G)这一步抽象成节点,特效师和算法研究团队就可以定制工作流和打磨最终产品,例如在P2G这一步后对网格进行风格化编辑,添加一个新的节点,引入额外力场,打磨艺术化效果。
\end{enumerate}


针对上述问题和挑战,本文旨在优化物质点法的内在缺陷,通过改造粒子网格间的传输方式解决非物理数值粘性问题,通过自适应网格划分更进一步节省计算开销并优化自碰撞边界空白缝隙问题,并设计与开发一套自研节点仿真系统实现本文提出的改进MPM算法,并在大规模流固耦合仿真场景中进行测试,做出算法效率、算法效果与制作流程便利性上的分析。

\section{国内外研究现状}
\subsection{流固耦合仿真}
在计算机图形学领域,流固耦合现象的仿真一直是研究的热点,固体模拟和流体模拟通常采用不同的离散方法和数值方法,如何正确处理交接处的数值解算是研究的难点。
Batty等~\cite{batty2007fast} 通过将压力求解表示为能量最小化问题,开发了一种方法,用于在笛卡尔欧拉流体与具有不规则几何形状的拉格朗日刚体之间进行双向耦合。Wolper等~\cite{klingner2006fluid} 提出了使用非结构化动态四面体网格来模拟流体的方法,该方法可以与拉格朗日刚体耦合。Chentanez~\cite{chentanez2006simultaneous} 公式化了一个单一的非对称线性系统,用于模拟流体与可变形固体之间的相互作用;该方法后来被 Zarifi~\cite{zarifi2017positive} 扩展为求解一个对称正定(SPD)系统。基于拉格朗日粒子的方法也能实现稳定的流固耦合。Akinci等~\cite{akinci2012versatile} 提出了一个基于流体动力学力的SPH流体与任意刚体耦合方法,该方法后来被扩展以支持弹性固体~\cite{akinci2013coupling}。

自FLIP方法被Zhu~\cite{zhu2005animating}引入图形学界以来,关于混合粒子-网格方法的研究取得了显著进展。Gao等~\cite{gao2009simulating} 提出了一个混合算法,通过结合网格和粒子方法高效地模拟气态流体,能够捕捉低速和高速行为。Losasso等~\cite{losasso2008two} 提出了一个双向耦合仿真框架,结合了用于密液体的粒子水平集方法和用于弥漫区的新型SPH方法,实现了密液体和弥漫区水体的高效建模,能够实现无缝混合和交互。Raveendran等~\cite{raveendran2011hybrid} 提出了一个混合方法,用于在SPH中强制保持不可压缩性。

\subsection{物质点法}
MPM是一种混合方法,通过使用拉格朗日粒子来承载材料信息,并使用欧拉网格进行力计算,从而统一了固体和流体的处理方法。最初为计算机图形学中的雪模拟所引入~\cite{stomakhin2013material},后来,MPM被扩展到处理多材料仿真中的相变问题~\cite{stomakhin2014augmented}。MPM能够处理多种材料,包括弹性-粘塑性体和非牛顿流体。Jiang等~\cite{jiang2015affine}提出了仿射粒子-网格转移方案,相比FLIP方法,减少了能量耗散,并提供了更高的稳定性。Klar等~\cite{klar2016drucker}在MPM中离散化了Drucker-Prager塑性模型来模拟沙流,并进一步扩展到沙水耦合模拟~\cite{tampubolon2017multi}。Jiang等~\cite{jiang2017anisotropic}还修改了MPM工作流,以支持布料模拟。Gao等~\cite{gao2017adaptive}开发了一个自适应MPM框架,用于子网格边界处理和改进的碰撞处理。MPM的一大显著优势是能够有效处理断裂现象,使其在涉及材料破裂和失效的仿真中尤为有用。Wolpher等~\cite{wolper2019cd}提出了连续损伤材料点方法,用于模拟包含大范围弹塑性变形的动态断裂。

MPM天生支持多种材料的耦合,并在网格上处理碰撞。然而,这也导致了数值粘性的相关问题。由于粒子在相同的B样条核宽度内难以相互分离,传统MPM中的流固界面通常具有无滑移边界条件。Han等~\cite{han2019hybrid} 将碰撞求解分为两个方面:超弹性能量模型,通过对碰撞施加惩罚,以及粒子-网格-粒子转移方案,类似于平滑核。许多研究者已经为解决这个问题做出了贡献。为不同材料采用不同网格是解决这一问题的常见方法。在追踪流固界面后,计算基于固相与两相之间网格速度差异的碰撞力,用于防止两相之间的粒子碰撞~\cite{yan2018mpm,han2019hybrid}。然而,这种方法仅限于显式时间积分,并且存在不稳定性问题。Fang等~\cite{fang2020iq} 提出了一种新颖的方案,使用两种不同的背景网格来模拟弹性固体与流体之间强耦合的相互作用,该方案采用了界面积分(IQ)离散化,用于实现自由滑移边界条件。在MPM中,也有采用单一网格的解决方案。Hu等~\cite{hu2018moving} 提出了CPIC转移方案,通过强制速度不连续性,支持更好的软体薄刚体耦合和流体薄刚体耦合的模拟。然而,该方法仅支持具有显式几何表示的刚体,无法模拟流体与弹性固体之间的相互作用,同时强制施加自由滑移条件。Fei等~\cite{fei2021revisiting} 基于NFLIP思想使用改进的转移方案来解锁MPM中的粒子。与传统的MPM积分器相比,该方案减少了扩散和非物理粘性,但如果位置调整参数选择不当,可能会导致不稳定性并容易出现粒子相交问题。

\subsection{自适应网格算法}
TODO

\section{研究目标和内容}
针对前文提到的问题与挑战,本文的研究目标将聚焦于设计一种全新的物质点法粒子网格传输算法,并结合自适应网格算法,以模拟流固耦合现象中流固接触与分离细节,并节省计算开销,同时设计与开发一款基于节点系统的物理仿真软件。结合国内外相关研究的成果,本文提出了一系列改进算法,其主要的研究贡献总结如下:

\begin{enumerate} 
    \item[1.] 针对传统MPM方法在统一背景网格上数值粘性的限制,本文通过修改传统的MPM流程,引入分解兼容仿射粒子网格(DC-APIC)积分器,在切向方向上强制实施自由滑移边界条件,在法向方向上实现分离条件,且通过多种复杂的流体与弹性固体相互作用的仿真进一步验证了该方法的有效性。为了寻找流固界面以及粒子与网格节点之间的兼容性,本文设计了一种新颖的基于相场梯度的分割方法,具有高精度和较高的计算效率。
    \item[2.] 针对传统MPM处理物体碰撞时会产生白色缝隙的问题。本文提出一种融合了分解兼容仿射粒子网格(DC-APIC)积分器的自适应网格算法,可以在不带来大量额外开销都同时有效保留不同尺度的碰撞细节。
    \item[3.] 设计并实现基于节点系统的物理仿真软件,其中提供了一套程序化编辑和运算的系统,支持模型导入、物理仿真、光追渲染等功能,并且在该软件框架内部实现了本文的物质点法改进算法。
 \end{enumerate}
 围绕这些研究内容,本文设置了多个具有挑战性的实验场景,例如海啸摧毁城市、不同密度橡胶玩具落水、溃坝和软体交互等,验证本文在产生真实感视觉效果方面的鲁棒性和实用性,同时还设置多个验证对比实验进行具体分析。总而言之,本文成功的提供了一个能够正确模拟流体与固体双向耦合交界面正确分离的仿真方法以及一套节点化物理仿真软件。

\section{论文组织结构}
TODO