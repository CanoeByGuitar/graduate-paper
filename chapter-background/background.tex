\chapter{相关概念及技术} \label{chap:background}


\section{本构模型}\label{sect:background:constitutive_model}
在MPM中,材料通过一组粒子表示,这些粒子携带材料的属性信息,如质量、速度和位置。与其他PIC/FLIP求解器类似,MPM在背景欧拉网格上求解控制方程。基于应力的力在网格上计算,并随后转移回粒子,粒子以拉格朗日方式进行运动。

MPM的形变理论来源于连续介质力学。在未形变的空间中,材料粒子的位置为$\mathbf{X}$,被映射到形变后的空间中,位置为$\mathbf{x}$,即$\mathbf{x} = \phi(\mathbf{X})$,其中$\phi$是形变映射。形变梯度$\mathbf{F}$是$\phi$的雅可比矩阵,表示为$\mathbf{F}=\frac{\partial \phi}{\partial \mathbf{X}}(\mathbf{X},t)$。控制方程由质量守恒和动量守恒组成:
\begin{equation}
\frac{D\rho}{Dt}+\rho \nabla \cdot \mathbf{v} = 0, \;\;
\rho \frac{D \mathbf{v}}{Dt} = \nabla \cdot\mathbf{\sigma} + \mathbf{\rho} \mathbf{f}_{ext},
\end{equation}
其中$\rho$是密度,$\mathbf{v}$是速度,$\mathbf{\sigma}$是Cauchy应力张量,$\mathbf{f}_{ext}$是外力,如重力。形变梯度可以分解为$\mathbf{F}=\mathbf{F^E}\mathbf{F^P}$,其中$\mathbf{F^E}$和$\mathbf{F^P}$分别表示弹性部分和塑性部分。应力和应变可以表示为:
\begin{equation}
    \mathbf{\sigma}=\frac{1}{J}\frac{\partial \mathbf{\psi}}{\partial \mathbf{F}}, \; 
    J = det(F),
\end{equation}
其中$J$是$\mathbf{F}$的行列式,表示材料粒子的体积变化。
MPM可以通过使用本构关系来模拟各种材料,本构关系描述了应力和应变之间的关系。本文中,我们使用Neo-Hookean模型~\cite{DBLP:journals/tog/TuLZLWWQ24}来推导塑性和流体材料。本构能量$\psi$被分解为体积部分$\psi_{vol}$和偏差部分$\psi_{dev}$:
\begin{equation}
    \begin{aligned}
        \psi_{dev} &= \frac{\mu}{2}(J^{-2/d} tr(\mathbf{F}\mathbf{F}^T) - d), \\
        \psi_{vol} &= \frac{\kappa}{2}(
        \frac{J^2-1}{2}-log(J)
        ),
    \end{aligned}
\end{equation}
其中$d$是维度,$\mu$是剪切模量,$\kappa$是体积模量。然后应力可以表示为:
\begin{equation}
\begin{aligned}
    \mathbf{\tau} &= \frac{1}{J}\mathbf{\sigma},\\
    \mathbf{\tau} &= \mathbf{\tau}_{vol}+\mathbf{\tau}_{dev},\\
    \mathbf{\tau}_{vol} &= \frac{\kappa}{2}(J^2-1)\mathbf{I}+\mu dev(J^{-2/d} \mathbf{F}\mathbf{F}^T),
\end{aligned}
\end{equation}
其中$\tau$是Kirchhoff应力,$dev(\mathbf{A})=\mathbf{A}-\frac{1}{d}tr(\mathbf{A})\mathbf{I}$表示任意张量$\mathbf{A}$的偏差部分。

大多数材料在剪切应力超过特定屈服条件时也表现出塑性变形,可以通过屈服函数$y$来描述。任何违反此条件的应力(即$y > 0$)应通过回归映射过程投影到屈服面上。本文中,我们采用了Drucker-Prager模型~\cite{klar2016drucker}来描述颗粒材料和流体。为避免体积增益伪影,我们通过追踪和恢复体积变化,使用改进的回归映射算法~\cite{10.1145/3072959.3073651}。


\section{物质点法}
\subsection{求解步骤}
TODO
\subsection{粒子网格传输算法}
在MPM中,传输方案是指在每个仿真步骤中,将物理量从材料点传递到欧拉网格的过程。由于MPM结合了拉格朗日和欧拉框架的优点,因此在这两个领域之间的传输对准确建模材料行为至关重要。

在以下上下文中,上标$n$、$n+1$表示时间$t^n$和$t^{n+1}$下的物理量,为了简便,我们设定$\Delta t=t^{n+1}-t^n$,$\Delta \mathbf{x}_{ip}^{n}=\mathbf{x}_i-\mathbf{x}_p^n$,并且$w_{ip}=N_i(\mathbf{x}_p)$,其中$N_i(\mathbf{x}_p)$表示节点$i$和粒子$p$之间的二次B样条插值函数,主要用于传输步骤。

标准的PIC方案根据以下公式将质量$m_p$、位置$\mathbf{x}_p$和速度$\mathbf{v}_p$从粒子传输到欧拉网格:
\begin{equation}
    \begin{aligned}
    P2G: \;\;    
    &m_i\mathbf{v}_i^n =\sum_{p}w_{ip}m_p\mathbf{v}_p^n, \\
    G2P:\;\; & \mathbf{v}_p^{n+1}=\sum_iw_{ip}\mathbf{v}_i^*,\\
    &\mathbf{x}_p^{n+1}=\mathbf{x}_p^n+\Delta t\sum_iw_{ip}\mathbf{v^*}
    \end{aligned}
\end{equation}
其中$\mathbf{v}_i^*$表示在应力和碰撞解决后更新的网格速度。PIC传输方案稳定,但存在过度的耗散问题。

FLIP传输方案\cite{zhu2005animating}通过添加一个额外的粒子速度保持项来减轻PIC中的耗散,表达为:
\begin{equation}
    \begin{aligned}
    P2G: \;\;    
    &m_i\mathbf{v}_i^n =\sum_{p}w_{ip}m_p\mathbf{v}_p^n, \\
    G2P:\;\; & \mathbf{v}_p^{n+1}=\sum_iw_{ip}\mathbf{v}_i^* + \alpha(\mathbf{v}_p^n-\sum_iw_{ip}\mathbf{v}_i^n),\\
    &\mathbf{x}_p^{n+1}=\mathbf{x}_p^n+\Delta t\sum_iw_{ip}\mathbf{v^*},
    \end{aligned}
    \label{eq:flip}
\end{equation}
其中$\alpha$是PIC-FLIP混合比率~\cite{bridson2015fluid}。当$\alpha=0$时,公式~\eqref{eq:flip}回归为PIC,当$\alpha=1$时,它是完全的FLIP。尽管FLIP传输方案通过额外的粒子速度保持项使仿真更加生动,但它仍然存在不稳定性。

APIC \cite{jiang2015affine}方案通过保持角动量和防止不稳定性来减少一部分耗散。我们总结了APIC的P2G和G2P传输步骤如下:
\begin{equation}
    \begin{aligned}
    P2G: \;\;    
    &m_i\mathbf{v}_i^n =\sum_{p}w_{ip}m_p(\mathbf{v}_p^n + \mathbf{B}_p^n(\mathbf{D}_p^n)^{-1}\Delta\mathbf{x}_{ip}), \\
    G2P:\;\; & \mathbf{v}_p^{n+1}=\sum_iw_{ip}\mathbf{v}_i^*,\\
    &\mathbf{B}_p^{n+1} = \sum_{i}w_{ip}\mathbf{v}_i^*\Delta\mathbf{x}_{ip},\\
    &\mathbf{x}_p^{n+1}=\mathbf{x}_p^n+\Delta t\sum_iw_{ip}\mathbf{v^*},
    \end{aligned}
\end{equation}
其中$\mathbf{D}_p^n$类似于惯性张量,表示为
\begin{equation}
    \mathbf{D}_p^n=\sum_{i}w_{ip}\Delta\mathbf{x}_{ip}(\Delta\mathbf{x}_{ip})^T
\end{equation}
局部速度仿射项$\mathbf{B}_p\mathbf{D}_p^{-1}$有助于保持能量,否则这种能量会因PIC等传输方案而丧失,尤其是由于数值粘性。然而,APIC仍然基于材料连续性的假设,因此无法有效处理粒子分离。

Affine-augmented Separable FLIP(ASFLIP)方案\cite{fei2021revisiting}旨在解决材料连续性假设可能不成立的情况,例如对于高度分散的沙子。它在FLIP中添加了粒子速度保持项,并在APIC中添加了一个额外的位置调整项,以便更容易地实现粒子分离:
\begin{equation}
    \begin{aligned}
    P2G: \;\;    
    &m_i\mathbf{v}_i^n =\sum_{p}w_{ip}m_p(\mathbf{v}_p^n + \mathbf{B}_p^n(\mathbf{D}_p^n)^{-1}\Delta\mathbf{x}_{ip}), \\
    G2P:\;\; & \mathbf{v}_p^{n+1}=\sum_iw_{ip}\mathbf{v}_i^*+ \alpha(\mathbf{v}_p^n-\sum_iw_{ip}\mathbf{v}_i^n),\\
    &\mathbf{B}_p^{n+1} = \sum_{i}w_{ip}\mathbf{v}_i^*\Delta\mathbf{x}_{ip},\\
    &\mathbf{x}_p^{n+1}=\mathbf{x}_p^n+\Delta t(\sum_iw_{ip}\mathbf{v^*}+ \beta_p\alpha(\mathbf{v}_p^n-\sum_iw_{ip}\mathbf{v}_i^n)),
    \end{aligned}
\end{equation}
其中$\beta_p$是控制位置调整幅度的比例。然而,它仍然存在数值粘性问题。这个问题在流体-固体耦合问题中尤为突出,在这些问题中,固体和流体都被离散化为MPM粒子,使得两个相位之间难以平滑分离。此外,ASFLIP的效果受到参数的影响,这使得它更容易产生穿透伪影,如图~\ref{fig:asflip}所示。

对于混合粒子-网格方法,数值粘性来自粒子和网格之间的传输。通常,粒子的数量远大于网格节点的数量。因此,P2G步骤可能导致信息丢失。如图~\ref{fig:p2g}(a)所示,当两个质量相等且速度相反的粒子将动量传输到网格时,速度将被平均化。从另一个角度来看,MPM基于材料连续性的假设,这使得它难以在物理上描述材料分离,特别是在流体-固体耦合的情况下。两个相位之间的界面速度场很容易被P2G步骤平滑化。
